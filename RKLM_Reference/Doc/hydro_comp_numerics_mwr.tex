%%%%%%%%%%%%%%%%%%%%%%%%%%%%%%%%%%%%%%%%%%%%%%%%%%%%%%%%%%%%%%%%%%%%%%
% hydro_comp_numerics_mwr.tex --  LaTeX-based template for submissions to American 
% Meteorological Society journals
%
% Template developed by Amy Hendrickson, 2013, TeXnology Inc., 
% amyh@texnology.com, http://www.texnology.com
% following earlier work by Brian Papa, American Meteorological Society
%
% Email questions to latex@ametsoc.org.
%
%%%%%%%%%%%%%%%%%%%%%%%%%%%%%%%%%%%%%%%%%%%%%%%%%%%%%%%%%%%%%%%%%%%%%
% PREAMBLE
%%%%%%%%%%%%%%%%%%%%%%%%%%%%%%%%%%%%%%%%%%%%%%%%%%%%%%%%%%%%%%%%%%%%%

%% Start with one of the following:
% DOUBLE-SPACED VERSION FOR SUBMISSION TO THE AMS
%\documentclass{ametsoc}

% TWO-COLUMN JOURNAL PAGE LAYOUT---FOR AUTHOR USE ONLY
 \documentclass[twocol]{ametsoc}

%%%%%%%%%%%%%%%%%%%%%%%%%%%%%%%%
%%% To be entered only if twocol option is used

\journal{mwr}

%  Please choose a journal abbreviation to use above from the following list:
% 
%   jamc     (Journal of Applied Meteorology and Climatology)
%   jtech     (Journal of Atmospheric and Oceanic Technology)
%   jhm      (Journal of Hydrometeorology)
%   jpo     (Journal of Physical Oceanography)
%   jas      (Journal of Atmospheric Sciences)	
%   jcli      (Journal of Climate)
%   mwr      (Monthly Weather Review)
%   wcas      (Weather, Climate, and Society)
%   waf       (Weather and Forecasting)
%   bams (Bulletin of the American Meteorological Society)
%   ei    (Earth Interactions)

%%%%%%%%%%%%%%%%%%%%%%%%%%%%%%%%
%Citations should be of the form ``author year''  not ``author, year''
\bibpunct{(}{)}{;}{a}{}{,}

%%%%%%%%%%%%%%%%%%%%%%%%%%%%%%%%

\usepackage{amsmath}
\usepackage{amssymb}
\usepackage{amsthm}

%\usepackage[squaren]{SIunits}
\usepackage{graphicx}

%----------------------------------------------------------------------------
%---- Farben und Macros zur Textmarkierung ----------------------------------
%----------------------------------------------------------------------------
\usepackage{color}
\definecolor{light}{gray}{0.50}
\definecolor{heavy}{gray}{0.35}
\definecolor{black}{gray}{0.0}
\definecolor{dgreen}{rgb}{0.0,0.7,0}
\definecolor{dred}{rgb}{0.9959,0,0}
\definecolor{green}{rgb}{0.0,0.99599,0.0}
\definecolor{purple}{rgb}{0.6,0.0,0.4}

\newcommand{\red}[1]{\textcolor{dred}{#1}}
\newcommand{\green}[1]{\textcolor{green}{#1}}
\newcommand{\dgreen}[1]{\textcolor{dgreen}{#1}}
\newcommand{\purple}[1]{\textcolor{purple}{#1}}
\newcommand{\blue}[1]{\textcolor{blue}{#1}}
\newcommand{\black}[1]{\textcolor{black}{#1}}
\newcommand{\grey}[1]{\textcolor{heavy}{#1}}
\newcommand{\lightgrey}[1]{\textcolor{light}{#1}}

\theoremstyle{definition}
\newtheorem{algorithm}{Algorithm}


%----------------------------------------------------------------------------
%---- Kommentare ------------------------------------------------------------
%----------------------------------------------------------------------------

\newcommand{\klein}[1]{\textcolor{blue}{#1}}
\newcounter{kleincommentno}
\setcounter{kleincommentno}{1}
\newcommand{\kleincomment}[1]{{\small\bfseries\textcolor{blue}{ {\small\bfseries${}^{[\arabic{kleincommentno}]}$}}%
\marginpar{\textcolor{blue}{{\small{\bfseries[\arabic{kleincommentno}]}\ \small #1}}\addtocounter{kleincommentno}{1}}}}

\newcommand{\benacchio}[1]{\textcolor{red}{#1}}
\newcounter{benacchiocommentno}
\setcounter{benacchiocommentno}{1}
\newcommand{\benacchiocomment}[1]{{\small\bfseries\textcolor{red}{ {\small\bfseries${}^{[\arabic{benacchiocommentno}]}$}}%
\marginpar{\textcolor{red}{{\small{\bfseries[\arabic{benacchiocommentno}]}\ \small #1}}\addtocounter{benacchiocommentno}{1}}}}

%----------------------------------------------------------------------------
%---- Eigene Macros ---------------------------------------------------------
%----------------------------------------------------------------------------
\newcommand{\hsc}{h_{\rm sc}}
\newcommand{\order}[1]{^{(#1)}}
\let\dss=\displaystyle

\renewcommand{\vector}[1]{\relax\ifmmode\mathchoice
{\mbox{\boldmath$\displaystyle#1$}}
{\mbox{\boldmath$\displaystyle#1$}}
{\mbox{\boldmath$\scriptstyle#1$}}
{\mbox{\boldmath$\scriptscriptstyle#1$}}\else
\hbox{\boldmath$\textstyle#1$}\fi}

\newcommand{\eq}[1]{(\ref{#1})}

\newcommand{\cbar}{\overline{c}}
\newcommand{\thetabar}{\overline{\theta}}
\newcommand{\thetatilde}{\widetilde{\theta}}

\newcommand{\vk}{\vector{k}}
\newcommand{\vu}{\vector{u}}
\newcommand{\vv}{\vector{v}}
\newcommand{\vx}{\vector{x}}

\newcommand{\ubar}{\overline{u}}
\newcommand{\vbar}{\overline{v}}

\newcommand{\advection}[1]{\mathcal{A}\left[#1\right]}
\newcommand{\advectionNum}[1]{\widetilde{\mathcal{A}}\left[#1\right]}
\newcommand{\rhs}[1]{\mathcal{R}\left[#1\right]}
\newcommand{\source}[1]{S\left(#1\right)}
\newcommand{\Id}{{\rm Id}}

\newcommand{\pprime}{p'}
\newcommand{\half}{\frac{1}{2}}
\newcommand{\quart}{\frac{1}{4}}
\newcommand{\eightth}{\frac{1}{8}}

\newcommand{\ie}{\emph{i.e.}}

\newcommand{\Nsq}{N^2}
\newcommand{\Nscsq}{(\tau N)^2}

\newcommand{\dt}{\Delta t}
\newcommand{\dz}{\Delta z}
\newcommand{\chibar}{\overline{\chi}}
\newcommand{\chitilde}{{\widetilde \chi}}
\newcommand{\Thetabar}{\overline{\Theta}}
\newcommand{\Thetatilde}{{\widetilde \Theta}}
\newcommand{\pibar}{\overline{\pi}}
\newcommand{\pitilde}{{\widetilde \pi}}

\newcommand{\pp}[2]{\frac{\partial #1}{\partial #2}}
\newcommand{\ppn}[3]{\frac{\partial^{#1} #2}{\partial #3^{#1}}}

\newcommand{\bigoh}[1]{\mathcal{O}\left(#1\right)}
\newcommand{\littleoh}[1]{{\scriptstyle\mathcal{O}}\left(#1\right)}

\newcommand{\reals}{\mathbb{R}}

\newcommand{\eps}{\varepsilon}
\newcommand{\rbeta}{\red{\vector{\beta}}}
\newcommand{\balpha}{\blue{\vector{\alpha}}}

\newcommand{\Lopfirst}{{\cal L}^{\rm 1st}}



%%% To be entered by author:

%% May use \\ to break lines in title:

\title{A semi-implicit numerical model for small-to-planetary scale atmospheric dynamics}

%%% Enter authors' names, as you see in this example:
%%% Use \correspondingauthor{} and \thanks{Current Affiliation:...}
%%% immediately following the appropriate author.
%%%
%%% Note that the \correspondingauthor{} command is NECESSARY.
%%% The \thanks{} commands are OPTIONAL.

\authors{Tommaso Benacchio\correspondingauthor{Tommaso  Benacchio, MOX - Modelling and Scientific Computing,
Dipartimento di Matematica, Politecnico di Milano, via Bonardi 9, 20133 Milano, Italy}}

\email{tommaso.benacchio@polimi.it}

\affiliation{MOX - Modelling and Scientific Computing,
Dipartimento di Matematica, Politecnico di Milano, via Bonardi 9, 20133 Milano, Italy}

\extraauthor{Rupert Klein}

\extraaffil{FB Mathematik \& Informatik, Freie Universit\"at Berlin, Germany}

%%%%%%%%%%%%%%%%%%%%%%%%%%%%%%%%%%%%%%%%%%%%%%%%%%%%%%%%%%%%%%%%%%%%%
% ABSTRACT
%
% Enter your Abstract here

\abstract{} 

\begin{document}

%% Necessary!
\maketitle


%%%%%%%%%%%%%%%%%%%%%%%%%%%%%%%%%%%%%%%%%%%%%%%%%%%%%%%%%%%%%%%%%%%%%
% MAIN BODY OF PAPER
%%%%%%%%%%%%%%%%%%%%%%%%%%%%%%%%%%%%%%%%%%%%%%%%%%%%%%%%%%%%%%%%%%%%%
%
\section{Introduction}

\section{Methods}

\section{Numerical Results}

\subsection{Vortex}

\cite{KadiogluEtAl2008}

\subsection{Acoustic Wave}

\subsection{Density current}

\subsection{Straka}

\subsection{Inertia-gravity wave} 

Inertia-gravity wave of \cite{SkamarockKlemp1994}, non-hydrostatic, hydrostatic, and (NEW) planetary scale

\subsubsection{Nonhydrostatic case}

\subsubsection{Hydrostatic case}

\subsubsection{Planetary-scale case}

\subsubsection{Multiscale case}

\cite{BaldaufBrdar2013}


 Run some of the tests with prescribed and dynamically computed 
      background states.
      
 Influence of reduction of advection accuracy to first order in the
      advective flux predictor


\section{Discussion and conclusion}

\acknowledgments

%%%%%%%%%%%%%%%%%%%%%%%%%%%%%%%%%%%%%%%%%%%%%%%%%%%%%%%%%%%%%%%%%%%%%
% REFERENCES
%%%%%%%%%%%%%%%%%%%%%%%%%%%%%%%%%%%%%%%%%%%%%%%%%%%%%%%%%%%%%%%%%%%%%

\bibliographystyle{ametsoc2014}
\bibliography{Bibliography}


%%%%%%%%%%%%%%%%%%%%%%%%%%%%%%%%%%%%%%%%%%%%%%%%%%%%%%%%%%%%%%%%%%%%%
% FIGURES---PLACE AT END OF DOCUMENT
%%%%%%%%%%%%%%%%%%%%%%%%%%%%%%%%%%%%%%%%%%%%%%%%%%%%%%%%%%%%%%%%%%%%%
%\begin{figure}[h]
% \centerline{\includegraphics[width=19pc]{figure01.pdf}}
%  \caption{Enter the caption for your figure here.  Repeat as
%  necessary for each of your figures. Figure from \protect\cite{Knutti2008}.}\label{f1}
%\end{figure}
%

%%%%%%%%%%%%%%%%%%%%%%%%%%%%%%%%%%%%%%%%%%%%%%%%%%%%%%%%%%%%%%%%%%%%%
% TABLES---PLACE AT END OF DOCUMENT
%%%%%%%%%%%%%%%%%%%%%%%%%%%%%%%%%%%%%%%%%%%%%%%%%%%%%%%%%%%%%%%%%%%%%
%\begin{table}[h]
%\caption{This is a sample table caption and table layout.  
%Table from Lorenz (1963).}\label{t1}
%\begin{center}
%\begin{tabular}{ccccrrcrc}
%\topline
%$N$ & $X$ & $Y$ & $Z$\\
%\midline
% 0000 & 0000 & 0010 & 0000 \\
% 0005 & 0004 & 0012 & 0000 \\
% 0010 & 0009 & 0020 & 0000 \\
% 0015 & 0016 & 0036 & 0002 \\
% 0020 & 0030 & 0066 & 0007 \\
% 0025 & 0054 & 0115 & 0024 \\
%\botline
%\end{tabular}
%\end{center}
%\end{table}
%%


%%%%%%%%%%%%%%%%%%%%%%%%%%%%%%%%%%%%%%%%%%%%%%%%%%%%%%%%%%%%%%%%%%%%%
% APPENDIXES
%%%%%%%%%%%%%%%%%%%%%%%%%%%%%%%%%%%%%%%%%%%%%%%%%%%%%%%%%%%%%%%%%%%%%

%% If only one appendix, use
%%\appendix%

%% If more than one appendix, use \appendix[<letter>], e.g.,
% \appendix[A] 

%\appendixtitle{Title of Appendix}


%\subsection{Appendix section}

%%%%%%%%%%%%%%%%%%%%%%%%%%%%%%%%%%%
%APPENDIX FIGURE AND TABLE EXAMPLES---PLACE AT END OF DOCUMENT
%%%%%%%%%%%%%%%%%%%%%%%%%%%%%%%%%%%
%
%\begin{table}
%\appendcaption{A1}{Here is the appendix table caption.}
%\centering
%\begin{tabular}{ccc}
%a&b&c\\
%d&e&f
%\end{tabular}
%\end{table}
%
%\begin{figure}
%\centerline{(illustration here)}
%\appendcaption{A1}{Here is the appendix figure caption.}
%\end{figure}



\end{document}
%%%%%%%%%%%%%%%%%%%%%%%%%%%%%%%%%%%%%%%%%%%%%%%%%%%%%%%%%%%%%%%%%%%%%
% END OF HYDRO_COMP_NUMERICS_MWR.TEX
%%%%%%%%%%%%%%%%%%%%%%%%%%%%%%%%%%%%%%%%%%%%%%%%%%%%%%%%%%%%%%%%%%%%%