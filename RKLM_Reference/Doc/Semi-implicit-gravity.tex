\documentclass[11pt,a4]{article}

\parindent=0pt
\parskip=3pt

\setlength{\paperheight}{29.5cm}
\setlength{\paperwidth}{21.2cm}

%\setlength{\voffset}{-2.0cm}
\setlength{\headheight}{0cm}
\setlength{\headsep}{0cm}
\setlength{\textheight}{23.7cm}
\setlength{\textwidth}{16cm}
\setlength{\oddsidemargin}{0cm}
\setlength{\evensidemargin}{0cm}
\setlength{\topmargin}{0cm}
\setlength{\topskip}{0cm}

\catcode`\"=\active \let"=\"
\let\3=\ss

\usepackage[latin9]{inputenc}

\usepackage[sort]{natbib}
\bibpunct{(}{)}{,}{a}{}{,}

\usepackage{amsmath}
\usepackage{amssymb}
\usepackage{amsthm}
\usepackage{IEEEtrantools}

%\usepackage[squaren]{SIunits}
\usepackage{graphicx}

%----------------------------------------------------------------------------
%---- Farben und Macros zur Textmarkierung ----------------------------------
%----------------------------------------------------------------------------
\usepackage{color}
\definecolor{light}{gray}{0.50}
\definecolor{heavy}{gray}{0.35}
\definecolor{black}{gray}{0.0}
\definecolor{dgreen}{rgb}{0.0,0.7,0}
\definecolor{dred}{rgb}{0.9959,0,0}
\definecolor{green}{rgb}{0.0,0.99599,0.0}
\definecolor{purple}{rgb}{0.6,0.0,0.4}

\newcommand{\red}[1]{\textcolor{dred}{#1}}
\newcommand{\green}[1]{\textcolor{green}{#1}}
\newcommand{\dgreen}[1]{\textcolor{dgreen}{#1}}
\newcommand{\purple}[1]{\textcolor{purple}{#1}}
\newcommand{\blue}[1]{\textcolor{blue}{#1}}
\newcommand{\black}[1]{\textcolor{black}{#1}}
\newcommand{\grey}[1]{\textcolor{heavy}{#1}}
\newcommand{\lightgrey}[1]{\textcolor{light}{#1}}

\theoremstyle{definition}
\newtheorem{algorithm}{Algorithm}


%----------------------------------------------------------------------------
%---- Kommentare ------------------------------------------------------------
%----------------------------------------------------------------------------

\newcommand{\klein}[1]{\textcolor{blue}{#1}}
\newcounter{kleincommentno}
\setcounter{kleincommentno}{1}
\newcommand{\kleincomment}[1]{{\small\bfseries\textcolor{blue}{ {\small\bfseries${}^{[\arabic{kleincommentno}]}$}}%
\marginpar{\textcolor{blue}{{\small{\bfseries[\arabic{kleincommentno}]}\ \small #1}}\addtocounter{kleincommentno}{1}}}}

\newcommand{\benacchio}[1]{\textcolor{red}{#1}}
\newcounter{benacchiocommentno}
\setcounter{benacchiocommentno}{1}
\newcommand{\benacchiocomment}[1]{{\small\bfseries\textcolor{red}{ {\small\bfseries${}^{[\arabic{benacchiocommentno}]}$}}%
\marginpar{\textcolor{red}{{\small{\bfseries[\arabic{benacchiocommentno}]}\ \small #1}}\addtocounter{benacchiocommentno}{1}}}}

%----------------------------------------------------------------------------
%---- Eigene Macros ---------------------------------------------------------
%----------------------------------------------------------------------------
\newcommand{\hsc}{h_{\rm sc}}
\newcommand{\order}[1]{^{(#1)}}
\let\dss=\displaystyle

\renewcommand{\vector}[1]{\relax\ifmmode\mathchoice
{\mbox{\boldmath$\displaystyle#1$}}
{\mbox{\boldmath$\displaystyle#1$}}
{\mbox{\boldmath$\scriptstyle#1$}}
{\mbox{\boldmath$\scriptscriptstyle#1$}}\else
\hbox{\boldmath$\textstyle#1$}\fi}

\newcommand{\eq}[1]{(\ref{#1})}

\newcommand{\cbar}{\overline{c}}
\newcommand{\thetabar}{\overline{\theta}}
\newcommand{\thetatilde}{\widetilde{\theta}}

\newcommand{\vk}{\vector{k}}
\newcommand{\vu}{\vector{u}}
\newcommand{\vv}{\vector{v}}
\newcommand{\vx}{\vector{x}}

\newcommand{\vV}{\vector{V}}

\newcommand{\ubar}{\overline{u}}
\newcommand{\vbar}{\overline{v}}

\newcommand{\advection}[1]{\mathcal{A}\left[#1\right]}
\newcommand{\advectionNum}[1]{\widetilde{\mathcal{A}}\left[#1\right]}
\newcommand{\rhs}[1]{\mathcal{R}\left[#1\right]}
\newcommand{\source}[1]{S\left(#1\right)}
\newcommand{\Id}{{\rm Id}}

\newcommand{\pprime}{p'}
\newcommand{\half}{\frac{1}{2}}
\newcommand{\quart}{\frac{1}{4}}
\newcommand{\eightth}{\frac{1}{8}}

\newcommand{\ie}{\emph{i.e.}}

\newcommand{\Nsq}{N^2}
\newcommand{\Nscsq}{(\tau N)^2}

\newcommand{\dt}{\Delta t}
\newcommand{\dz}{\Delta z}
\newcommand{\chibar}{\overline{\chi}}
\newcommand{\chitilde}{{\widetilde \chi}}
\newcommand{\Thetabar}{\overline{\Theta}}
\newcommand{\Thetatilde}{{\widetilde \Theta}}
\newcommand{\pibar}{\overline{\pi}}
\newcommand{\pitilde}{{\widetilde \pi}}

\newcommand{\pp}[2]{\frac{\partial #1}{\partial #2}}
\newcommand{\ppn}[3]{\frac{\partial^{#1} #2}{\partial #3^{#1}}}

\newcommand{\bigoh}[1]{\mathcal{O}\left(#1\right)}
\newcommand{\littleoh}[1]{{\scriptstyle\mathcal{O}}\left(#1\right)}

\newcommand{\reals}{\mathbb{R}}

\newcommand{\eps}{\varepsilon}
\newcommand{\rbeta}{\red{\vector{\beta}}}
\newcommand{\balpha}{\blue{\vector{\alpha}}}

\newcommand{\Lopfirst}{{\cal L}^{\rm 1st}}

\newcommand{\Sol}{U}

\newcommand{\rfr}[1]{#1_{\text{ref}}}

\newcommand{\nablatilde}{{\widetilde\nabla}}

% ===========================================================================
% ===========================================================================
% ===========================================================================
% ===========================================================================

\title{``Catchy Name''�-- Semi-implicit gravity}
\author{T.~Benacchio, MetOffice, Exeter, UK\\ 
        R.~Klein, Mathematics \& Informatics, Freie Universit\"at Berlin}

\begin{document}

\maketitle


% ===========================================================================
% ===========================================================================
% ===========================================================================

\section{Introduction}
\label{sec:Intro}

% ===========================================================================
% ===========================================================================

\subsection{Governing equations}
\label{ssec:GoverningEquations}

The governing equations for 
the adiabatic compressible model version and for an inert ideal gas with 
constant specific heat capacities under the influence of gravity and in 
a rotating coordinate system corresponding to a tangent plane approximation
may be written as
%
\begin{subequations}\label{eq:CompressibleEuler}
\begin{IEEEeqnarray}{rCl}
\label{eq:EulerMass}
\dss \rho_t + \nabla_\parallel\cdot(\rho \vu) + (\rho w)_z
  & = 
    & \dss 0
      \\[5pt]
\label{eq:EulerHorMom}
\dss (\rho\vu)_t + \nabla_\parallel\cdot(\rho \vu\circ\vu) + (\rho w \vu)_z 
  & = 
    & \dss - \left( c_p  P \nabla_\parallel \pi + f(y) \vk \times \rho\vu \right)
      \\[5pt]
\label{eq:EulerVerMom}
\dss (\rho w)_t + \nabla_\parallel\cdot(\rho \vu w) + (\rho w^2)_z 
  & = 
    & \dss - \left(  c_p P \pi_z + \rho g \right)
      \\[5pt]
\label{eq:EulerPressure}
\dss P_t + \nabla_\parallel\cdot(P\vu) + (Pw)_z
  & = 
    & \dss 0\,.
\end{IEEEeqnarray}
\end{subequations}
%
Here $\rho$ is the density, $\vu, w$ are the horizontal and vertical components 
of the flow velocity, 
%
\begin{equation}\label{eq:EOSpiP}
\pi = \left(\frac{p}{\rfr{p}}\right)^{\frac{R}{c_p}}
\qquad\text{and}\qquad
P = \frac{\rfr{p}}{R} \left(\frac{p}{\rfr{p}}\right)^{\frac{c_v}{c_p}} \equiv \rho\Theta
\end{equation}
%
are the Exner pressure and the density weighted potential temperature, respectively, 
with $\rfr{p}$ a suitable reference pressure, $R$ the gas constant and $c_p$ and $c_v$ the 
specific heat capacities at constant pressure and constant volume. Furthermore, $g$ is the
(constant) acceleration of gravity, $f(y) = f_0 + \beta y$ the local Coriolis parameter in 
the ``$\beta$-plane'' with suitable constants $f_0$ and $\beta$, and $\vk$ the vertical 
unit vector. 

Given \eq{eq:EulerMass} and \eq{eq:EulerPressure}, the potential temperature
$\Theta = P/\rho$ satisfies the usual advection equation
%
\begin{equation}
\Theta_t + \vu\cdot\nabla_\parallel \Theta + w \Theta_z = 0\,.
\end{equation}

% ===========================================================================
% ===========================================================================

\subsection{Perturbation vs.\ full variables and conservation form}
\label{ssec:PertFullCons}

\klein{Here we should point out difficulties others have had in formulating
a scheme that systematically advances full variables, and in formulating
a scheme in conservation form that is implicit w.r.t.\ buoyancs (e.g.,
Hilary's MWR paper). What else ... ?}

% ===========================================================================
% ===========================================================================

\subsection{EULAG: Implicit trapezoidal rule applied along Lagrangian trajectories}
\label{ssec:EULAG}

\klein{Not sure we will need this section in this form, but find it helpful 
to have a summary of the EULAG strategy written down somewhere.}

The semi-implicit time integration procedure implemented in EULAG, 
\citep{PrusaEtAl2008}, \citep{SmolarkiewiczEtAl2014}, relies on introducing
a perturbation potential temperature via
%
\begin{equation}\label{eq:ThetaSplit}
\Theta(t,\vx,z) = \Thetatilde(t,\vx,z) + \Thetabar(z)
\end{equation}
%
and reformulating \eq{eq:CompressibleEuler} as
%
\begin{IEEEeqnarray}{rCrCl}\label{eq:EULAGCompressibleEuler}
\dss \rho_t 
  & + 
    & \dss \nabla_\parallel\cdot(\rho\vu) + (\rho w)_z \hfil
      & = 
        & \dss 0
          \IEEEyesnumber\IEEEyessubnumber*\label{eq:EULAGRhoEqn}\\[5pt]
\dss (\rho\vu)_t 
  & + 
    & \dss \nabla_\parallel\cdot(\rho\vu\circ\vu) + (\rho w\,\vu)_z  \hfil
      & = 
        & \dss - \left( c_p \rho\Theta\nabla_\parallel \pi + f(y) \vk\times\rho\vu\right)
          \label{eq:EULAGHorMomEqn}\\[5pt]
\dss (\rho w)_t 
  & + 
    & \dss \nabla_\parallel\cdot(\rho\vu\, w) + (\rho w\, w)_z \hfil
      & = 
        & \dss - \left( c_p  \rho\Theta \pi_z + \rho g\right)
          \\[5pt]
\dss (\rho\Thetatilde)_t 
  & + 
    & \dss \nabla_\parallel\cdot(\rho\vu\, \Thetatilde) + (\rho w\, \Thetatilde)_z \hfil
      & = 
        & \dss -\left[\nabla_\parallel\cdot(\rho\vu\, \Thetabar) + (\rho w\, \Thetabar)_z\right]\,.
        \label{eq:EULAGThetaTildeEqn}
\end{IEEEeqnarray}
%
In a compact notation, these equations combine the mass balance from \eq{eq:EULAGRhoEqn} with
%
\begin{IEEEeqnarray}{rCl}\label{eq:RhoBasedAdvection}
(\rho\Phi)_t + \mathcal{A}(\Phi; \rho\vv) 
  & = 
    & R(\Phi; \rho)\, ,
\end{IEEEeqnarray}
%
where
%
\begin{equation}\label{eq:EULAGPhiDefinition}
\Phi = (\vu, w, \Thetatilde)^t\, ,
\end{equation}
%
and where $\mathcal{A}(\Phi; \rho\vv)$ denotes the advection terms on the left of
\eq{eq:EULAGHorMomEqn}--\eq{eq:EULAGThetaTildeEqn}, while and $R(\Phi, \rho)$ 
collects all terms on the right. 

Based on this notation, an update of the solution from time $t^n$ to time $t^{n+1}$ according
to EULAG's non-oscillatory forward-in-time integration strategy then involves first the
determination of mass flux densities at the half time level via
%
\begin{equation}\label{eq:EULAGFluxExtrapolation}
(\rho\vv)^{n+\half} = \frac{3}{2} (\rho\vv)^{n} - \frac{1}{2}  (\rho\vv)^{n-1}
\end{equation}
%
and the computation of the new time level density via
%
\begin{equation}
\rho^{n+1} 
= 
\rho^{n} + \Delta t\, \nablatilde \cdot (\rho\vv)^{n+\half}\,,
\end{equation}
%
where ``$\nablatilde\cdot\, $'' denotes a cell centered second-order approximation of the
divergence. Secondly, the vector of specific variables $\Phi$ from \eq{eq:EULAGPhiDefinition}
is updated through the following three-step procedure, 
%
\begin{IEEEeqnarray}{rCl}\label{eq:EULAGTimeIntegrator}
\dss (\rho\Phi)^{*} 
  & = 
    & \dss (\rho\Phi)^{n} + \frac{\Delta t}{2} R\left(\Phi^n; \rho^n\right)
      \IEEEyesnumber\IEEEyessubnumber*\label{eq:EULAGExplicitEulerStep}\\
\dss (\rho\Phi)^{**} 
  & = 
    & \dss \mathcal{A}_{\text{\tiny MPDATA}}^{\Delta t}\left(\Phi^*; (\rho\vv)^{n+1/2}\right)
      \label{eq:EULAGAdvectionStep}\\
\dss (\rho\Phi)^{n+1} 
  & = 
    & \dss (\rho\Phi)^{**} + \frac{\Delta t}{2} R\left(\Phi^{n+1}; \rho^{n+1}\right)\,.
      \label{eq:EULAGImplicitEulerStep}
\end{IEEEeqnarray}
%
Here $\mathcal{A}_{\text{\tiny MPDATA}}^{\Delta t}\left(\Phi; \vV\right)$ denotes
a time step by the ``Multidimensional Positive Definite Advection Transport 
Algorithm'' (MPDATA) \citep[see, e.g.,][]{SmolarkiewiczMPDATA2006}, which is a robust 
second-order accurate explicit approximation of the \emph{linear} 
advection of $\Phi$ by the \emph{given} advecting fluxes $\vV$ 
over a time increment $\Delta t$, i.e., a solution over a half 
time step of 
%
\begin{IEEEeqnarray}{rCl}
\dss (\rho\Phi)_t + \nabla\cdot(\vV\Phi)
  & =
    & 0\,.
\end{IEEEeqnarray}
%

We note in passing that the default option in EULAG for determining the advecting mass 
fluxes $(\rho\vv)^{n+\half}$, i.e., the extrapolation scheme from 
\eq{eq:EULAGFluxExtrapolation}, makes the method a three time level�scheme. 
An alternative two time level variant of EULAG instead carries out an intermediate, at 
least first order accurate, time step of the full equations from $t^{n}$ to $t^{n+1/2}$ and
thus generates $(\rho\vv)^{n+\half}$ without a need to recur to the $t^{n-1}$ 
level solution. To maintain the large time step stability of the overall semi-implicit
scheme, however, this alternative comes with the cost of an additional elliptic solve for
an intermediate pressure variable per time step. 

Following \citet{SmolarkiewiczMargolin1997}, the scheme in 
\eq{eq:EULAGTimeIntegrator} may be interpreted as the application of the implicit 
trapezoidal rule for time integrating the compatibility condition  
$D\Phi/Dt \equiv (\partial_t + \vv\cdot\nabla) \Phi = R(\Phi; \rho)/\rho$ 
that holds along Lagrangian characteristics of \eq{eq:EULAGCompressibleEuler}.  
Thus, a forward Euler step is followed by a backward Euler step, with the advection 
step interleaved to capture the advancement of the Lagrangian paths. 

An observation that may in part explain the outstanding large time step 
stability features of EULAG is that for given advecting flux $\vV = (\rho\vv)^{n+\half}$ 
the advection step is \emph{linear}. As a consequence, even if for rather large time 
steps the velocities computed in the forward Euler step \eq{eq:EULAGExplicitEulerStep} 
end up far away from the solutions at time levels $n$ or $n+1$ for large $\Delta t$, 
these excursions do (i) not affect the advecting fluxes $(\rho\vv)^{n+1/2}$, and they 
can (ii) properly be post-corrected and pulled back to the reasonable range by the 
backward Euler step in \eq{eq:EULAGImplicitEulerStep}. 

% ===========================================================================
% ===========================================================================
% ===========================================================================

\section{Summary of the discretization}
\label{sec:DiscretizationSummary}

% ===========================================================================
% ===========================================================================

\subsection{Reformulation of the governing equations}
\label{ssec:Reformulation}

\cite{BenacchioEtAl2014} suggested a seamless ``blended'' discretization of the 
compressible and pseudo-incompressible \citep{Durran1989} Euler equations. Following
\citet{KleinTCFD2009,KleinEtAl2010}, they observed that the pseudo-incompressible
model is obtained from the compressible equations in \eq{eq:CompressibleEuler} 
by simply dropping the time derivative of $P = \rho\Theta$ from \eq{eq:EulerPressure}.
To take advantage of this simple structural model relationship, they introduced 
the inverse of the potential temperature,
%
\begin{equation}
\chi = 1/\Theta\,,
\end{equation}
% 
and recast the mass balance \eq{eq:EulerMass} as a transport equation for $\chi$, 
%
\begin{equation}\label{eq:chiI}
\rho_t + \nabla_\parallel\cdot(\rho \vu) + (\rho w)_z = 
(P\chi)_t + \nabla_\parallel\cdot(P\chi \vu) + (P\chi w)_z = 0\,,
\end{equation}
%
in which the field $(P\vv)$ now takes the role of the ``advecting flux''. 
Using this interpretation consistently throughout the equation system we obtain, 
%
\begin{IEEEeqnarray}{rCrCl}\label{eq:EulerP}
\dss \rho_t 
  & + 
    & \dss \nabla_\parallel\cdot(P\vu\, \chi) + (Pw\, \chi)_z \hfil
      & = 
        & \dss 0
          \IEEEyesnumber\IEEEyessubnumber*\label{eq:EulerPMass}\\[5pt]
\dss (\rho\vu)_t 
  & + 
    & \dss \nabla_\parallel\cdot(P\vu\circ\chi\vu) + (Pw\, \chi\vu)_z  \hfil
      & = 
        & \dss - \left( c_p P\nabla_\parallel \pi + f(y) \vk\times\rho\vu\right)
          \label{eq:EulerPHorMom}\\[5pt]
\dss (\rho w)_t 
  & + 
    & \dss \nabla_\parallel\cdot(P \vu\, \chi w) + (Pw\, \chi w)_z \hfil
      & = 
        & \dss - \left( c_p P \pi_z + \rho g\right)
          \label{eq:EulerPVerMom}\\[5pt]
P_t
  &  +
    & \dss \dss \nabla_\parallel\cdot(P\vu)  + (Pw)_z  \hfil
      & = 
        & \dss 0\,,
        \label{eq:EulerPP}
\end{IEEEeqnarray}
%
and this formulation is the basis for the discretization proposed below. 

In formulating a scheme that is implicit with respect to buoyancy and potential temperature
advection, it is crucial to have separate access to a large-scale mean background stratification
of potential temperature, or of its inverse, and to its local perturbations. Thus, in analogy
with \eq{eq:ThetaSplit} we split $\chi$ into
%
\begin{equation}\label{eq:ChiSplit}
\chi(t,\vx,z) = \chitilde(t,\vx,z) + \chibar(z)
\end{equation}
% 
and obtain a perturbation version of the mass balance as
%
\begin{IEEEeqnarray}{rCrCl}\label{eq:EulerPChiTilde}
\dss (P\chitilde)_t 
  & + 
    & \dss \nabla_\parallel\cdot(P\vu\, \chitilde) + (Pw\, \chitilde)_z \hfil
      & = 
        & \dss -\left[\nabla_\parallel\cdot(P\vu\, \chibar) + (Pw\, \chibar)_z\right]\,.
        \label{eq:ChiTildeEqn}
\end{IEEEeqnarray}
%
In formulating the semi-implicit discretization, we will carry $\chitilde$ as an
auxiliary variable through a time step and utilize the full and perturbation versions
of the mass balance in \eq{eq:EulerPMass} and \eq{eq:EulerPChiTilde} alternatively 
depending on the context. At the end of each time step, the variables are synchronized 
again to match with the computed full density, $\rho$, and internal energy variable, $P$. 

In analogy with the split of the Euler equations into the mass transport and
forced advection equations in section~\ref{ssec:EULAG}, see \eq{eq:RhoBasedAdvection}, 
we split the augmented equations from \eq{eq:EulerP} and \eq{eq:EulerPChiTilde}
into the internal energy equation, \eq{eq:EulerPP}, and a set of forced advection 
equations
%
\begin{equation}\label{eq:ForcedPAdvection}
(P\Psi)_t + \mathcal{A}(\Psi; P\vv) = Q(\Psi; P)\, ,
\end{equation}
%
where
%
\begin{equation}\label{eq:PsiDefinition}
\Psi = (\chi, \chi\vu, \chi w, \chitilde)\,,
\end{equation}
% 
and where advection is now mediated by the $P$-fluxes,  
%
\begin{equation}
P\vv \equiv (P\vu, Pw)\,.
\end{equation}
% 


% ===========================================================================
% ===========================================================================

\subsection{Semi-implicit time discretization for compressible flow}
\label{ssec:TimeDiscretizationOverview}

% ===========================================================================

\subsubsection{Advective fluxes at the half time level}
\label{sssec:AdvectiveFluxes}

In analogy with EULAG's forward-in-time integration strategy, the first step
in our scheme is the determination of at least first order accurate advective 
fluxes $(P\vv)^{n+\half}$ at the half time level 
$t^{n+\half} = (t^{n} + t^{n+1})/2$. This is achieved applying standard splitting
into advective and non-advective terms in \eq{eq:EulerP}, \eq{eq:EulerPChiTilde}. 
Thus we first advance the advected scalars $\Psi$ from \eq{eq:PsiDefinition} by
half a time step utilizing advective fluxes computed at the old time level, 
%
\begin{IEEEeqnarray}{rCl}\label{eq:EULAGHalfTimePredictorAdvection}
\dss (P\Psi)^{\#} 
  & = 
    & \dss \mathcal{A}_{1\text{st}}^{\frac{\Delta t}{2}}\left(\Psi^{n}; (P\vv)^{n}\right)
      \label{eq:EULAGHalfTimePredictorAdvection}\,.
\end{IEEEeqnarray}
%
Here $\mathcal{A}_{1\text{st}}^{\Delta t}$ denotes an at least first order
accurate version of our advection scheme (see section~\ref{ssec:Advection}).  
Then, the half time level fluxes $(P\vv)^{n+\half}$ are obtained as a result
of an implicit Euler discretization of the full system,\kleincomment{double-check notation when details are written up.} 
%
\begin{IEEEeqnarray}{rCl}\label{eq:HalfTimePredictorFluxCorrection}
\dss (P\Psi)^{n+\half} 
  & = 
    & \dss (P\Psi)^{\#} + \frac{\Delta t}{2} Q\left(\Psi^{n+1/2}; P^{n+\half}\right)\,,
      \label{eq:HalfTimePredictorRHS}
      \IEEEyesnumber\IEEEyessubnumber*\\
\dss P^{n+\half} 
  & = 
    & \dss P^{n} - \frac{\Delta t}{2} \nabla\cdot(P\vv)^{n+\half}\,.
      \label{eq:HalfTimePredictorRHS}
\end{IEEEeqnarray}
%
This may or may not be combined with a linearization of the equations of state 
\eq{eq:EOSpiP} which relate $P$ and $\pi$, i.e.,  
%     
\begin{IEEEeqnarray}{rCl}\label{eq:EULAGHalfTimePredictorPLinearization}
\dss P^{n+\half} 
  & = 
    & \dss P^{n} 
      + \left(\frac{\partial P}{\partial \pi}\right)^{\#} 
        \left(\pi^{n+\half} - \pi^{n}\right)\,.
\end{IEEEeqnarray}
%
With the linearization, the implicit step involves a single linear elliptic solve
for $\pi^{n+\half}$, while without the linearization an outer (Newton) iteration 
must be invoked to guarantee consistency with the equation of state up to the 
desired accuracy. 

These preliminary calculations serve to provide the advecting fluxes 
$(P\vv)^{n+\half}$ subsequently needed in the advection step of the final
time stepping algorithm, \eq{eq:AdvectionStep}.

% ===========================================================================

\subsubsection{Full time step with given advective fluxes}
\label{sssec:FullTimeStep}

\klein{I am tempted to propose our scheme looks quite like a variant of EULAG, 
i.e., like \eq{eq:EULAGTimeIntegrator}, as spelled out in \eq{eq:TimeIntegrator}�
below.  Yet, I have my doubts regarding the 
treatment of the Exner pressure in the momentum equations. I am not sure the 
$P$-$\pi$ coupling through the equations of state is correctly captured in 
\eq{eq:ImplicitEulerStep}.
%
\begin{IEEEeqnarray}{rCl}\label{eq:TimeIntegrator}
\dss (P\Psi)^{*} 
  & = 
    & \dss (P\Psi)^{n} + \frac{\Delta t}{2} R\left(\Psi^n; P^n\right)
      \IEEEyesnumber\IEEEyessubnumber*\label{eq:ExplicitEulerStep}\\
\dss (P\Psi)^{**} 
  & = 
    & \dss \mathcal{A}_{2\text{\tiny nd}}^{\Delta t}\left(\Psi^*; (P\vv)^{n+1/2}\right)
      \label{eq:AdvectionStep}\\
\dss (P\Psi)^{n+1} 
  & = 
    & \dss (P\Psi)^{**} + \frac{\Delta t}{2} R\left(\Psi^{n+1}; P^{n+1}\right)\,.
      \label{eq:ImplicitEulerStep}
\end{IEEEeqnarray}
%
To sort this out, I guess I need to first spell out the details faithfully along the 
lines of the code and then go back an try to formulate the scheme in a compact notation.
}

% ===========================================================================
% ===========================================================================

\subsection{Cartesian cell centered grid arrangement}
\label{ssec:GridArrangement}

In this paper we describe our scheme for two-dimensional flows. Extension 
to three space dimensions is straightforward and implemented in our 
research code. The scheme is fully cell centered on control volumes 
$C_{i,j}$ formed by a cartesian mesh, with constant grid increments 
$\Delta x, \Delta y$, and grid indices $i = 0, ..., I-1$ and 
$j = 0, ..., J-1$. The discrete approximate solution involves averages of 
the solution variables over grid cells, i.e., 
%
\begin{equation}
\Sol_{i,j}^n \approx 
\frac{1}{\Delta x \Delta y}\int\limits_{C_{i,j}} \Sol(\vx,t^n)\, d^2\vx
\qquad\text{where}\qquad 
\Sol = \left(
\begin{array}{c}
\rho \\ \rho\vu \\ \rho w \\ P
\end{array}
\right)\,.
\end{equation}
%
The scheme is second order accurate, so that -- within the approximation
order -- we can interchangeably interpret $\Sol_{i,j}^n$ as the cell average, 
as above, or as a point value of $\Sol$ at the center of mass of the grid 
cell.  

% ===========================================================================
% ===========================================================================

\subsection{Synchronization of auxiliary variables}
\label{ssec:Synchronization}

The pressure $p$ plays a multiple roles in the compressible Euler equations
as formulated in \eq{eq:CompressibleEuler}. The cell-centered volume average of 
$P = (\rfr{p}/R) (p/\rfr{p})^{c_v/c_p}$ is representative of the internal energy
budget of the flow, whereas the gradient of the Exner pressure 
$\pi = (p/\rfr{p})^{R/c_p}$ arises in the coupling of the momentum field to
the pressure. As a consequence, in controlling both the momenta $(\rho \vv)$ 
at the cell centers and the advective fluxes $(P\vv)$ at cell faces through 
the implicit Euler steps in \eq{eq:HalfTimePredictorFluxCorrection} and 
\eq{eq:TimeIntegrator}, respectively, we will need both 
node and cell centered approximations of the Exner pressure (see 
sections~\ref{ssec:DivControlledAdvectiveFluxes} and 
\ref{ssec:ExplicitImplicitEuler}).

These auxiliary fields are synchronized at the end of, say, the $n$th time
step with the primary cell centered variable $P^n_{i,j}$ through
%
\begin{IEEEeqnarray}{rCl}
\pi^{n}_{i,j} 
  & =
    & \left(\frac{RP}{\rfr{p}}\right)^{\frac{R}{c_v}}
      \IEEEyesnumber\IEEEyessubnumber*\\[10pt]
\pi^{n}_{i+\half,j+\half}
  & =
    & \frac{1}{4} 
      \left(\pi^{n}_{i,j} + \pi^{n}_{i+1,j} + \pi^{n}_{i,j+1} + \pi^{n}_{i+1,j+1} 
      \right)\,.
\end{IEEEeqnarray}
%

Similarly, the split of the full (inverse of) the potential temperature 
$\chi = \chitilde + \chibar$ and the subsequent separate updates of  
$\chi$ and $\chitilde$ require re-synchronization. Thus, after completion
of a time step we let
%
\begin{equation}
\chitilde^{n} = \chi^{n} - \chibar^{n}\,,
\end{equation}   
%
where in all subsequent simulations we have frozen $\chibar^{n} \equiv \chibar^{0}$. 
An alternative option better suited for large scale, long time simulations
would be to invoke a horizontal, possibly local, averaging procedure to extract
$\chibar$ from $\chi$ at least every few time steps. We leave testing this option
to future work.	 


% ===========================================================================
% ===========================================================================

\subsection{Advection}
\label{ssec:Advection}

Any robust numerical scheme capable of performing advection of a scalar in
compressible flows is a good candidate for the generic discrete
advection operator $\mathcal{A}^{\Delta t}$ introduced above. EULAG uses 
variants of the ``multi-dimensional positive definite advection transport 
algorithm'' (MPDATA) \citet{PrusaEtAl2008}. The present implementation involves
a directionally split ``monotone upwind scheme for conservation laws'' (MUSCL): 

Suppose the half time predictor step from \eq{eq:EULAGHalfTimePredictor}, 
the details of which will be described shortly, has been completed. Then, 
the advecting flux in the $x$ direction is approximated as, 
%
\begin{IEEEeqnarray}{rCl}
(Pu)^{n+\half}_{i+\half,j} 
  & = 
    & \frac{1}{2}\left((Pu)^{n+\half}_{i,j} + (Pu)^{n+\half}_{i+1,j}\right)
      \IEEEyesnumber\IEEEyessubnumber*\\
(Pv)^{n+\half}_{i,j+\half} 
  & = 
    & \frac{1}{2}\left((Pv)^{n+\half}_{i,j} + (Pv)^{n+\half}_{i,j+1}\right)\,.
\end{IEEEeqnarray}
%
Given these fluxes, the advection step in \eq{eq:EULAGAdvectionStep} is discretized 
via Strang splitting, so that
%
\begin{equation}\label{eq:AdvectionStrangSplitting}
\Sol_{i,j}^{**} 
=
\mathcal{A}_{2\text{nd}}^{\Delta t} \Sol_{i,j}^{*} 
\equiv
\mathcal{A}^{\frac{\Delta t}{2}}_x 
\mathcal{A}^{\frac{\Delta t}{2}}_y 
\mathcal{A}^{\frac{\Delta t}{2}}_y 
\mathcal{A}^{\frac{\Delta t}{2}}_x\, \Sol_{i,j}^{*} \,,
\end{equation}
%
where
%
\begin{equation}
\mathcal{A}^{\frac{\Delta t}{2}}_x\, \Sol_{i,j} 
= \Sol_{i,j}
- \frac{\Delta t}{2\Delta x} 
  \left((Pu)^{n+\half}_{i+\half,j}\, \Psi_{i+\half,j} 
      - (Pu)^{n+\half}_{i-\half,j}\, \Psi_{i-\half,j} \right)
\end{equation}
%
with 
%
\begin{IEEEeqnarray}{rCl}\label{eq:AdvSpecifics}
\Psi_{i+\half,j} 
  & = 
    & \sigma \Psi^{-}_{i+\half,j} + (1-\sigma)\Psi^{+}_{i+\half,j}\,,
      \IEEEyesnumber\IEEEyessubnumber*\\[8pt]
\sigma 
  & = 
    & \text{sign}\left((Pu)^{n+\half}_{i+\half,j}\right)\,,
      \\[8pt]
\Psi^{-}_{i+\half,j} 
  & = 
    & \Psi_{i,j} + \frac{\Delta x}{2} \left(1 - C^{n+\half}_{i+\half,j} \right)  s_{i,j}\,,
      \\[8pt]
\Psi^{+}_{i+\half,j} 
  & = 
    & \Psi_{i+1,j} - \frac{\Delta x}{2} \left(1 + C^{n+\half}_{i+\half,j} \right)  s_{i+1,j}\,,
      \\[8pt]
C^{n+\half}_{i+\half,j}
  & =
    & \frac{\Delta t}{\Delta x} \frac{(Pu)^{n+\half}_{i+\half,j}}{(P_{i,j} + P_{i+1,j})/2}\,,
      \label{eq:AdvSpecificsCourantNo}\\[8pt]
s_{i,j}
  & =
    & \text{Lim}
      \left(\frac{U_{i,j}-U_{i-1,j}}{\Delta x}, \frac{U_{i+1,j}-U_{i,j}}{\Delta x}\right)\,,
      \label{eq:AdvSpecificsSlopes}
\end{IEEEeqnarray}
%
where $P_{i,j}$ in \eq{eq:AdvSpecificsCourantNo} denotes the fourth component 
of $U_{i,j}$, and $\text{Lim}(a,b)$ is a slope limiting function (see 
\citet{Sweby1984}). The effect of the slope limiting procedure on our simulation
results will be discussed in section~\ref{sec:Results} below. Analogous formulae 
hold for $\mathcal{A}^{\frac{\Delta t}{2}}_y U_{i,j}$. 

Importantly, the advecting fluxes 
$(Pu)^{n+\half}_{i+\half,j}, (Pv)^{n+\half}_{i,j+\half}$
are maintained unchanged throughout the Strang splitting cycle 
\eq{eq:AdvectionStrangSplitting}.

The first order accurate advection operator $\mathcal{A}_{1\text{st}}^{\Delta t}$
used in \eq{eq:EULAGHalfTimePredictorAdvection} is a simplified version
of the above in that (i) the advective fluxes are approximated at the old time level
%
\begin{IEEEeqnarray}{rCl}
(Pu)^{n}_{i+\half,j} 
  & = 
    & \frac{1}{2}\left((Pu)^{n}_{i,j} + (Pu)^{n}_{i+1,j}\right)
      \IEEEyesnumber\IEEEyessubnumber*\\
(Pv)^{n}_{i,j+\half} 
  & = 
    & \frac{1}{2}\left((Pv)^{n}_{i,j} + (Pv)^{n}_{i,j+1}\right)
\end{IEEEeqnarray}
%
and (ii) we use straightforward sequential instead of Strang splitting, i.e.,  
%
\begin{equation}
\mathcal{A}_{1\text{st}}^{\frac{\Delta t}{2}}
=
\mathcal{A}^{\frac{\Delta t}{2}}_x 
\mathcal{A}^{\frac{\Delta t}{2}}_y 
\qquad\text{and}\qquad
s_{i,j} \equiv 0\,,
\end{equation}
%
in our standard configuration. We will demonstrate below that such a 
reduction to first order accuracy in the predictor step does not only not
affect the formal order of consistency of the scheme, but also hardly 
influences the solution quality at finite resolution.\kleincomment{t.b.d.}

% ===========================================================================
% ===========================================================================

\subsection{Divergence controlled advective fluxes via eqn.\
\eq{eq:EULAGHalfTimePredictorRHS}}
\label{ssec:DivControlledAdvectiveFluxes}


% ===========================================================================

\subsection{Explicit \eq{eq:EULAGExplicitEulerStep} and implicit  
\eq{eq:EULAGImplicitEulerStep} Euler steps for the generalized sources}
\label{ssec:ExplicitImplicitEuler}

In solving \eq{eq:EULAGHalfTimePredictorRHS}, the goal is to obtain first
order accurate approximations of the advecting fluxes 
$(Pu)^{n+\half}_{i+\half,j}$ and $(Pv)^{n+\half}_{i,j+\half}$ which are 
located at the grid cell faces. Having completed the advection predictor, 
\eq{eq:EULAGHalfTimePredictorAdvection}, we first generate face centered 
approximations
%
\begin{IEEEeqnarray}{rCl}
(Pu)^{\#}_{i+\half,j} 
  & = 
    & (Pu)^{\#}_{i,j} + (Pu)^{\#}_{i+1,j}
      \IEEEyesnumber\IEEEyessubnumber*\\
(Pu)^{\#}_{i,j+\half} 
  & = 
    & (Pv)^{\#}_{i,j} + (Pu)^{\#}_{i,j+1}
\end{IEEEeqnarray}
%


% ===========================================================================

\subsection{Generalized source terms for the cell centered momenta}
\label{ssec:GeneralizedSourceTermsCellCenteredMomenta}


% ===========================================================================
% ===========================================================================
% ===========================================================================

\section{Results}
\label{sec:Results}

% ===========================================================================
% ===========================================================================
% ===========================================================================

\section{Conclusions}
\label{sec:Conclusions}

\begin{itemize}

\item differences relative to the compressible EULAG

\item differences to Hilary's scheme

\item differences to ENDGAME, GUNGHO and the like

\item differences to George Bryan's model

\item differences to COSMO, ICON etc. ... 

\item differences to ASAM

\item differences to Michael Dumbser's approach

\item what else?

\end{itemize}

% ===========================================================================
% ===========================================================================
% ===========================================================================

\bibliographystyle{FluidMechanics}
\bibliography{Bibliography}

\end{document}

% ===========================================================================
% ===========================================================================
% ===========================================================================
% ===========================================================================
% ===========================================================================
% ===========================================================================

\newpage

\hfil {\LARGE\bfseries Version of April 2018}


Hi Tommaso, \\

in dreaming up the semi-implicit gravity version in the appendix of your
thesis, I think we missed one crucial point. Reconsidering, I came to the
conclusion that it is the ``explicit part of the implicit step''�in the
trapezoidal rule that controls vertical velocities, and thus stability, 
for large time steps. I'll try to highlight this in the following short 
note. \\

Best, \\

Rupert

% ===========================================================================
% ===========================================================================
% ===========================================================================

\section{The second step of the implicit trapezoidal rule}
\label{sec:trapezoidal}

Focusing just on the crucial linear terms in the governing equations that
are responsible for internal waves, we have -- for advection of the background 
potential temperature and for the vertical momentum, 
%
\begin{eqnarray}
w_t 
  & = 
    & - \frac{\theta}{\Gamma}\left(\pi_z + \Gamma g \chi\right)
      \\
\chi_t
  & =
    & - w \frac{d\overline{\chi}}{dz}
\end{eqnarray}
%
here $\pi$ is Exner pressure, $\Gamma = (\gamma - 1) / \gamma$ with $\gamma$
the isentropic exponent, $\chi = 1/\theta$, and the prefactor $\theta$ in the 
first equation is frozen in for the linearization. Suppose that an explicit 
Euler forward step for these equations over $\Delta t / 2$ as well as 
advection of all quantities have been taken care of in operator splitting-like 
steps, so that all that is left to do is an implicit, backward Euler step over 
$\tau = \Delta t / 2$. 

Suppose that $(\chi^*, w^*, \pi^*)$ denote the state of the relevant solution 
variables at the start of the implicit substep and 
$(\delta\chi, \delta w, \delta\pi)$ the respective updates over the implicit
step. Then,
%
\begin{eqnarray}
\label{eq:deltaw1}
w^{n+1} - w^* \equiv \delta w 
  & = 
    & - \tau \, \frac{\theta}{\Gamma}\left([\pi^*+\delta\pi]_z + \Gamma g [\chi^* + \delta\chi]\right)
      \\
\label{eq:deltachi1}
\chi^{n+1} - \chi^* \equiv \delta\chi
  & =
    & - \tau\, (w^* + \delta w) \frac{d\overline{\chi}}{dz}\,.
\end{eqnarray}
%

Now we reorder explicit and implicit contributions, replace $\delta\chi$ in 
\eq{eq:deltaw1} using \eq{eq:deltachi1} and solve for $\delta w$, 
%
\begin{eqnarray}
w^{n+1} - w^* = \delta w 
  & = 
    & - \tau \, \frac{\theta}{\Gamma}\left(\pi^*_z + \Gamma g \chi^*\right)
      - \tau \, \frac{\theta}{\Gamma}\left(\delta\pi_z + \Gamma g \delta\chi\right)
      \\
  & =
    & - \tau \, \frac{\theta}{\Gamma}\left(\pi^*_z + \Gamma g \chi^*\right)
      - \tau \, \frac{\theta}{\Gamma}\left(\delta\pi_z 
      - \tau \, \Gamma g (w^* + \delta w) \frac{d\overline{\chi}}{dz}\right)
      \\
\delta w \left(1 - \tau^2 g \theta \frac{d\overline{\chi}}{dz}\right)
  & =
    & - \tau \, \frac{\theta}{\Gamma}\left(\pi^*_z + \Gamma g \chi^*\right)
      + w^* \tau^2 g \theta \frac{d\overline{\chi}}{dz}
      - \tau \, \frac{\theta}{\Gamma} \delta\pi_z \,.
\end{eqnarray}
%
Letting
%
\begin{equation}
\Nsq = - g \theta \frac{d\overline{\chi}}{dz}
\end{equation}
%
denote the square of the buoyancy-frequency, we have
%
\begin{equation}
\label{eq:deltaw2}
\delta w = \frac{1}{1 + \Nscsq} 
\left(- \tau \frac{\theta}{\Gamma}\left(\pi^*_z + \Gamma g \chi^*\right)
      - w^* \Nscsq\right)
      - \tau \frac{\theta/\Gamma}{1 + \Nscsq} \delta\pi_z
\end{equation}
%
and the update formular for $\chi$ subsequently follows from \eq{eq:deltachi1}.

Consider now the first term on the right in \eq{eq:deltaw2}, which contains all
explicit contributions to the velocity update, and its scaling for large time
steps $\tau \to \infty$. The first term in the bracket scales linearly with
$\tau$, so that it vanishes in the limit. The second term in the bracket involves
$(\tau N)^2$, however, and therefore the limit reads
%
\begin{equation}
\delta w\big|_{\tau \to \infty} = -w^*\, .
\end{equation}
%
That is, the first thing the explicit part of the implicit step does is to 
set the vertical velocity to zero in that limit! This is clearly a stabilizing
effect that can capture any large excursion in $w^*$ that may have incurred 
during the explicit half-time step of the trapezoidal rule. 

Tommaso, I think we left this bit out in our earlier attempts. I have implemented
a variant of this in my scheme and see the stabilization of vertical velocity 
very nicely for $N \Delta t$ as large as five. The large scale horizontal
domain with a hydrostatic gravity wave still does not quite work yet, but I have
two more things to try: (i) locally hydrostatic initialization of pressure - which
in my current setup I do not have, and (ii) a preconditioner that addresses the vertical
part of the Poisson problem, instead of preconditioning only through division by
the diagonal element of the Poisson-matrix. Will keep you posted. 

Best, 

Rupert

\newpage

% ===========================================================================
% ===========================================================================
% ===========================================================================

\section{Implicit trapezoidal rule -- variant 2}
\label{sec:trapezoidal2}

% ===========================================================================
% ===========================================================================

\subsection{Linear oscillator}

To understand the mechanism of stabilization for arbitrary time steps in the
implicit trapezoidal rule, we consider 
%
\begin{equation}
\ddot y + y = 0
\qquad\text{or rather}\qquad
\begin{array}{rcr}
\dot y 
  & = 
    & v
      \\
\dot v 
  & =
    & - y
\end{array}\,.
\end{equation}
%
The implicit trapezoidal rule for the system version of this problem over
one time step reads
%
\begin{equation}\label{eq:ImplicitTrapezoidalOscillator}
\begin{array}{rcr}
\dss y^{n+1} - y^n 
  & = 
    & \dss \frac{\dt}{2} \left(v^{n+1} +  v^n\right)
      \\[10pt]
\dss v^{n+1} - v^n 
  & =
    & \dss - \frac{\dt}{2} \left(y^{n+1} +  y^n\right)
\end{array}\,.
\end{equation}
%
We solve for $v^{n+1}$ first:
%
\begin{equation}\label{eq:ImplicitTrapezoidalStepOscillator}
\begin{array}{rcl}
\dss v^{n+1} 
  & = 
    & \dss v^n - \frac{\dt}{2} \left(y^n + \frac{\dt}{2}\left(v^{n+1} + v^n\right) + y^n\right) 
      \\[10pt]
\dss v^{n+1} \left(1 + \frac{(\dt)^2}{4}\right)
  & =
    & \dss v^n \left(1 - \frac{(\dt)^2}{4}\right) - \dt\, y^n
      \\
\dss v^{n+1} 
  & =
    & \dss \frac{1 - \frac{(\dt)^2}{4}}{1 + \frac{(\dt)^2}{4}} \  v^n 
          - \frac{\dt}{1 + \frac{(\dt)^2}{4}}\, y^n
\end{array}
\end{equation}
or
%
\begin{equation}
v^{n+1} - v^n  \equiv \delta v
= - \ 2 v^n  \ \frac{\frac{(\dt)^2}{4}}{1 + \frac{(\dt)^2}{4}} 
          - \frac{\dt}{1 + \frac{(\dt)^2}{4}}\, y^n
\end{equation}
%
%
Similarly, 
%
\begin{equation}
\begin{array}{rcl}
\dss y^{n+1} 
  & = 
    & \dss y^n + \frac{\dt}{2} \left(v^n - \frac{\dt}{2}\left(y^{n+1} + y^n\right) + v^n\right) 
      \\[10pt]
\dss y^{n+1} \left(1 + \frac{(\dt)^2}{4}\right)
  & =
    & \dss y^n \left(1 - \frac{(\dt)^2}{4}\right) + \dt\, v^n
      \\
\dss y^{n+1} 
  & =
    & \dss \frac{1 - \frac{(\dt)^2}{4}}{1 + \frac{(\dt)^2}{4}} \  y^n 
          + \frac{\dt}{1 + \frac{(\dt)^2}{4}}\, v^n
\end{array}
\end{equation}
%


Clearly, 
%
\begin{equation}
\lim\limits_{\dt\to\infty} (y^{n+1}, n^{n+1}) = - (y^n, v^n)\, ,
\end{equation}
%
and this characterizes the energy-preserving, oscillatory nature, and
unconditional neutrality of the scheme in the large-time step limit. 

Can we understand the implicit trapezoidal rule as a two-step scheme 
involving a forward Euler predictor and a backward Euler corrector? 
Let's see.

Forward Euler predictor:
%
\begin{equation}
\begin{array}{rcr}
\dss y^* - y^n 
  & = 
    & \dss \frac{\dt}{2} \, v^n
      \\[10pt]
\dss v^* - v^n 
  & =
    & \dss - \frac{\dt}{2} \, y^n
\end{array}\,.
\end{equation}
%

Backward Euler corrector:
%
\begin{equation}
\begin{array}{rcr}
\dss y^{n+1} - y^*
  & = 
    & \dss \frac{\dt}{2} \, v^{n+1}
      \\[10pt]
\dss v^{n+1} - v^* 
  & =
    & \dss - \frac{\dt}{2} \, y^{n+1}
\end{array}\,.
\end{equation}
%
Obviously, adding the two equation sets we get back the implicit 
trapezoidal rule from \eq{eq:ImplicitTrapezoidalOscillator}. Solving for
the new time data, however, given the predicted ones, we obtain
%
\begin{equation}
\begin{array}{rcl}
\dss v^{n+1} 
  & = 
    & \dss v^* - \frac{\dt}{2} \left(y^* + \frac{\dt}{2}\, v^{n+1}\right) 
      \\[10pt]
\dss v^{n+1} \left(1 + \frac{(\dt)^2}{4}\right)
  & =
    & \dss v^* - \frac{\dt}{2}\, y^*
      \\
\dss v^{n+1} 
  & =
    & \dss \frac{1}{1 + \frac{(\dt)^2}{4}} \  v^* 
          - \frac{1}{2}\frac{\dt}{1 + \frac{(\dt)^2}{4}}\, y^*
\end{array}
\end{equation}
%
For the increment within the half timestep we have
%
\begin{equation}
v^{n+1} - v^* \equiv \delta v 
= - \ v^* \ \frac{\frac{(\dt)^2}{4}}{1 + \frac{(\dt)^2}{4}}  
          - \frac{1}{2}\frac{\dt}{1 + \frac{(\dt)^2}{4}}\, y^*
\end{equation}
%



To get back to the original formula from 
\eq{eq:ImplicitTrapezoidalStepOscillator} we re-insert the predictor 
step to obtain
%
\begin{equation}
\begin{array}{rcl}
\dss v^{n+1} 
  & =
    & \dss \frac{1}{1 + \frac{(\dt)^2}{4}} \  v^* 
          - \frac{1}{2}\frac{\dt}{1 + \frac{(\dt)^2}{4}}\, y^*
      \\[15pt]
  & =
    & \dss \frac{1}{1 + \frac{(\dt)^2}{4}} \left(v^n - \frac{\dt}{2} y^n\right) 
          - \frac{1}{2}\frac{\dt}{1 + \frac{(\dt)^2}{4}} \left(y^n + \frac{\dt}{2} v^n\right)
      \\[15pt]
  & =
    & \dss \frac{1 - \frac{(\dt)^2}{4}}{1 + \frac{(\dt)^2}{4}} \,  v^n
          - \frac{\dt}{1 + \frac{(\dt)^2}{4}}\, y^n
      \\
\end{array}
\end{equation}
%
And analogously for $y^{n+1}$, i.e., 
%
\begin{equation}
y^{n+1} = \frac{1 - \frac{(\dt)^2}{4}}{1 + \frac{(\dt)^2}{4}} \,  y^n
          + \frac{\dt}{1 + \frac{(\dt)^2}{4}}\, v^n
\end{equation}
%

We distinguish the cases of $(y^n, v^n) = \bigoh{1}$ and $y^n = 0$ or
$v^n = 0$ and consider the limit $\dt\to\infty$: 
%
\begin{enumerate}

\item Obviously, when $(y^n, v^n) = \bigoh{1}$, we have
%
\begin{equation}
\left.y^{n+1}\right|_{\dt\to\infty} = - y^n\,.
\qquad\text{and}\qquad
\left.v^{n+1}\right|_{\dt\to\infty} = - v^n
\end{equation}
%

\item However, if the respective initial datum is zero, the limiting behavior
changes as follows. Suppose $y^n = 0$, but $v^n \not= 0$. Then, 
%
\begin{equation}
y^{n+1} = \frac{\dt}{1 + \frac{(\dt)^2}{4}}\, v^n
\qquad\text{and}\qquad
v^{n+1} = \frac{1 - \frac{(\dt)^2}{4}}{1 + \frac{(\dt)^2}{4}} \,  v^n \to - v^n\,.
\end{equation}
%

\end{enumerate}
%


Now, can we also understand the implicit trapezoidal rule as a two-step scheme 
involving a first the backward and second the forward Euler step? 
Let's see.

Backward Euler step:
%
\begin{equation}
\begin{array}{rcr}
\dss y^{*} - y^n
  & = 
    & \dss \frac{\dt}{2} \, v^{*}
      \\[10pt]
\dss v^{*} - v^n 
  & =
    & \dss - \frac{\dt}{2} \, y^{*}
\end{array}\,.
\end{equation}
%

Forward Euler step:
%
\begin{equation}\label{eq:FowardEulerII}
\begin{array}{rcr}
\dss y^{n+1} - y^* 
  & = 
    & \dss \frac{\dt}{2} \, v^*
      \\[10pt]
\dss v^{n+1} - v^* 
  & =
    & \dss - \frac{\dt}{2} \, y^*
\end{array}\,.
\end{equation}
%
Solving for
the intermediate time data first we obtain
%
\begin{equation}
\begin{array}{rcl}
\dss v^{*} 
  & = 
    & \dss v^n - \frac{\dt}{2} y^* 
      \\
\dss y^{*}
  & = 
    & \dss y^n + \frac{\dt}{2} v^* = y^n + \frac{\dt}{2} \left(v^n - \frac{\dt}{2} y^*\right)
      \\[10pt]
\dss y^{*} \left(1 + \frac{(\dt)^2}{4}\right)
  & =
    & \dss y^n + \frac{\dt}{2}\, v^n
      \\
\dss y^{*} 
  & =
    & \dss \frac{1}{1 + \frac{(\dt)^2}{4}} \  
          \left(y^n + \frac{\dt}{2}\, v^n\right)
\end{array}
\end{equation}
%
and then, 
%
\begin{equation}
\begin{array}{rcl}
\dss v^{*} 
  & = 
    & \dss v^n - \frac{\dt}{2} y^* 
      \\
  & = 
    & \dss v^n - \frac{\frac{\dt}{2}}{1 + \frac{(\dt)^2}{4}} 
          \left(y^n + \frac{\dt}{2} v^n\right)
      \\[10pt]
  & =
    & \dss \frac{1}{1 + \frac{(\dt)^2}{4}} \left(v^n - \frac{\dt}{2} y^n\right)
\end{array}
\end{equation}
%
Next we insert into the forward Euler step in \eq{eq:FowardEulerII} to obtain,
%
\begin{equation}
\begin{array}{rcl}
\dss v^{n+1} 
  & = 
    & \dss v^* - \frac{\dt}{2} y^* 
      \\
  & = 
    & \dss \frac{1}{1 + \frac{(\dt)^2}{4}} 
          \left(v^n - \frac{\dt}{2} y^n  - \frac{\dt}{2}\left(y^n + \frac{\dt}{2}v^n\right)
          \right)
      \\[10pt]
  & =
    & \dss \frac{1 - \frac{(\dt)^2}{4}}{1 + \frac{(\dt)^2}{4}} v^n 
          - \frac{\dt}{1 + \frac{(\dt)^2}{4}} y^n
\end{array}
\end{equation}
%
and that is the same as found in \eq{eq:ImplicitTrapezoidalStepOscillator}.

% ===========================================================================

\subsubsection{Summary}

Euler forward over $\Delta t$:
%
\begin{equation}\label{eq:FowardEulerIII}
\begin{array}{rcr}
\dss y^{\rm new}  
  & = 
    & \dss y^{\rm old} + \dt \, v^{\rm old}
      \\[10pt]
\dss v^{\rm new}  
  & =
    & \dss v^{\rm old} - \dt \, y^{\rm old}
\end{array}\,.
\end{equation}
%

Euler backward over $\Delta t$:
%
\begin{equation}
\begin{array}{rcl}
\dss y^{\rm new} 
  & =
    & \dss \frac{1}{1 + (\dt)^2} \  
          \left(y^{\rm old} + \dt\, v^{\rm old}\right)
       \ = \ y^{\rm old} - \frac{(\dt)^2}{1 + (\dt)^2} y^{\rm old} + \frac{\dt}{1 + (\dt)^2} v^{\rm old}
      \\[15pt]
\dss v^{\rm new} 
  & =
    & \dss \frac{1}{1 + (\dt)^2} \  
          \left(v^{\rm old} - \dt\, y^{\rm old}\right)
       \ = \ v^{\rm old} - \frac{(\dt)^2}{1 + (\dt)^2} v^{\rm old} - \frac{\dt}{1 + (\dt)^2} y^{\rm old}
\end{array}
\end{equation}
%

Implicit trapezoidal rule over $\dt$:
%
\begin{equation}
\begin{array}{rcl}
\dss y^{\rm new} 
  & =
    & \dss \frac{1 - \frac{(\dt)^2}{4}}{1 + \frac{(\dt)^2}{4}} y^{\rm old} 
          + \frac{\dt}{1 + \frac{(\dt)^2}{4}} v^{\rm old}
       \ = \ y^{\rm old} - \frac{2 \frac{(\dt)^2}{4}}{1 + \frac{(\dt)^2}{4}} y^{\rm old} 
         + \frac{\dt}{1 + \frac{(\dt)^2}{4}} v^{\rm old}
      \\[15pt]
\dss v^{\rm new} 
  & =
    & \dss \frac{1 - \frac{(\dt)^2}{4}}{1 + \frac{(\dt)^2}{4}} v^{\rm old} 
          - \frac{\dt}{1 + \frac{(\dt)^2}{4}} y^{\rm old}
       \ = \ v^{\rm old} - \frac{2 \frac{(\dt)^2}{4}}{1 + \frac{(\dt)^2}{4}} v^{\rm old} 
         - \frac{\dt}{1 + \frac{(\dt)^2}{4}} y^{\rm old}
\end{array}
\end{equation}
%

% ===========================================================================
% ===========================================================================

\subsection{Implicit gravity}

% ===========================================================================

\subsubsection{Explicit part of the implicit trapezoidal rule}

Focusing just on the crucial linear terms in the governing equations that
are responsible for internal waves, we have -- for advection of the background 
potential temperature and for the vertical momentum, 
%
\begin{eqnarray}
w_t 
  & = 
    & - \frac{\theta}{\Gamma}\left(\pi_z + \Gamma g \chi\right)
      \\
\chi_t
  & =
    & - w \frac{d\overline{\chi}}{dz}
\end{eqnarray}
%
here $\pi$ is Exner pressure, $\Gamma = (\gamma - 1) / \gamma$ with $\gamma$
the isentropic exponent, $\chi = 1/\theta$, and the prefactor $\theta$ in the 
first equation is frozen in for the linearization. Suppose we wish to integrate
these equations using the implicit trapezoidal rule over a time step $\dt$. 
Then,
%
\begin{eqnarray}
\label{eq:deltaw1a}
w^{n+1} - w^n \equiv \delta w 
  & = 
    & - \dt \, \frac{\theta}{\Gamma}
     \left(\left[\pi^n+\frac{\delta\pi}{2}\right]_z + \Gamma g 
           \left[\chi^n + \frac{\delta\chi}{2}\right]\right)
      \\
\label{eq:deltachi1a}
\chi^{n+1} - \chi^n \equiv \delta\chi
  & =
    & - \dt\, \left(w^n + \frac{\delta w}{2}\right) \frac{d\overline{\chi}}{dz}\,.
\end{eqnarray}
%

Now we reorder explicit and implicit contributions, replace $\delta\chi$ in 
\eq{eq:deltaw1a} using \eq{eq:deltachi1a} and solve for $\delta w$, 
%
\begin{eqnarray}
w^{n+1} - w^n = \delta w 
  & = 
    & - \dt \, \frac{\theta}{\Gamma}\left(\pi^n_z + \Gamma g \chi^n\right)
      - \frac{\dt}{2} \, \frac{\theta}{\Gamma}\left(\delta\pi_z + \Gamma g \delta\chi\right)
      \\
  & =
    & - \dt \, \frac{\theta}{\Gamma}\left(\pi^n_z + \Gamma g \chi^n\right)
      - \frac{\dt}{2} \, \frac{\theta}{\Gamma}\left(\delta\pi_z 
      - \dt \, \Gamma g \left(w^n + \frac{\delta w}{2}\right) \frac{d\overline{\chi}}{dz}\right)
      \\
\delta w \left(1 - \frac{(\dt)^2}{4} g \theta \frac{d\overline{\chi}}{dz}\right)
  & =
    & - \dt \, \frac{\theta}{\Gamma}\left(\pi^n_z + \Gamma g \chi^n\right)
      + w^n \frac{(\dt)^2}{2} g \theta \frac{d\overline{\chi}}{dz}
      - \frac{\dt}{2} \, \frac{\theta}{\Gamma} \delta\pi_z \,.
\end{eqnarray}
%
Letting
%
\begin{equation}
\Nsq = - g \theta \frac{d\overline{\chi}}{dz}
\end{equation}
%
denote the square of the buoyancy-frequency, we have
%
\begin{equation}
\label{eq:deltaw2b}
\delta w = \frac{1}{1 + \left(\frac{N\dt}{2}\right)^2} 
\left(- \dt \frac{\theta}{\Gamma}\left(\pi^n_z + \Gamma g \chi^n\right)
      - 2\, w^n \left(\frac{N\dt}{2}\right)^2\right)
      - \frac{\dt}{2} \frac{\theta/\Gamma}{1 + \left(\frac{N\dt}{2}\right)^2} \delta\pi_z
\end{equation}
%
and the update formula for $\chi$ subsequently follows from \eq{eq:deltachi1a}. We summarize
these results in the following formulae for the new time level data, 
%
\begin{eqnarray}
\label{eq:deltaw3}
w^{n+1} 
  & = 
    & \frac{1 - \left(\frac{N\dt}{2}\right)^2}{1 + \left(\frac{N\dt}{2}\right)^2}\, w^n
-\frac{\dt}{1 + \left(\frac{N\dt}{2}\right)^2} 
\left( \frac{\theta}{\Gamma}\left(\pi^n_z + \Gamma g \chi^n\right)\right)
      - \frac{\dt}{2} \frac{\theta/\Gamma}{1 + \left(\frac{N\dt}{2}\right)^2} \delta\pi_z
\\
\label{eq:deltaChi3}
\chi^{n+1} 
  & =
    & \frac{1 - \left(\frac{N\dt}{2}\right)^2}{1 + \left(\frac{N\dt}{2}\right)^2}\, \chi^n
-\frac{\dt}{1 + \left(\frac{N\dt}{2}\right)^2} \frac{d\overline{\chi}}{dz} \, w^n\,.
\end{eqnarray}
%
It is also interesting to note the change in time of the momentum source term 
for frozen $\pi^n$ in the following fashion, 
%
\begin{equation}
\left(\chi + \frac{\pi_z}{\Gamma g}\right)^{n+1}
=
\frac{1 - \left(\frac{N\dt}{2}\right)^2}{1 + \left(\frac{N\dt}{2}\right)^2}\, 
\left(\chi + \frac{\pi_z}{\Gamma g}\right)^n
-\frac{\dt}{1 + \left(\frac{N\dt}{2}\right)^2} \frac{d\overline{\chi}}{dz} \, w^n\,.
\end{equation}
%


Consider now the first term on the right in \eq{eq:deltaw3}, which contains all
explicit contributions to the velocity update, and its scaling for large time
steps $\dt \to \infty$. The first term in the bracket scales linearly with
$\dt$, so that it vanishes in the limit. The second term in the bracket involves
$(\dt N)^2$, however, and therefore the limit reads
%
\begin{equation}
\delta^{\rm expl} w\big|_{\dt \to \infty} = - 2 w^n
\qquad\text{or}\qquad
w^{n+1} = - w^n\, .
\end{equation}
%
That is, the first thing the explicit part of the implicit step does is to 
impose a timestep--to--timestep oscillation as expected. This is the energy-preserving
property of the implicit trapezoidal rule in the linear case.

% ===========================================================================

\subsubsection{Application to advective flux calculations}
\label{sssec:AdvectiveFluxes}

Here we have to consider the joint evolution of $P$ and $P\chi = \rho$ in the
vertical split half time step, i.e.,
%
\begin{eqnarray}
\label{eq:LinearizedConservativeChiEqn}
(P\chi)_t
  & = 
    & - (P\chibar w)_z
      \\
(P)_t
  & = 
    & - (P w)_z
\end{eqnarray}
%
and utilize the implicit trapezoidal rule as worked out in \eq{eq:deltaw3} 
to determine the advecting fluxes $(Pw)$. Thus, 
%
\begin{eqnarray}
\label{eq:ConservativeChiUpdate1}
(P\chi)_j^{n+\half} - (P\chi)_j^n
  & = 
    & - \frac{\dt}{2\dz} \left( (P\chibar w)_{j+\half}^{n+\quart} - (P\chibar w)_{j-\half}^{n+\quart} \right)
      \\
P_j^{n+\half} - P_j^n
  & = 
    & - \frac{\dt}{2\dz} \left( (P w)_{j+\half}^{n+\quart} - (P w)_{j-\half}^{n+\quart} \right)
\end{eqnarray}
%
What I want to show is that the update $\chi_j^{n+\half}-\chi_j^n$ is as 
nicely controlled for large $\dt$ as is the nonconservative update from
\eq{eq:deltaChi3}.

Rewrite \eq{eq:ConservativeChiUpdate1} as
%
\begin{eqnarray}
P^{n+\quart*}_j \left(\chi_j^{n+\half} - \chi_j^{n}\right)
  & =  
    & - \left(P^{n+\half}_j - P^n_j\right) \chi_j^{n+\quart*}
      - \frac{\dt}{2\dz} 
        \left( (P\chibar w)_{j+\half}^{n+\quart} - (P\chibar w)_{j-\half}^{n+\quart} 
        \right)
      \\[5pt]
  & =  
    & - \frac{\dt}{2\dz} 
        \left( (P w)_{j+\half}^{n+\quart} (\chibar_{j+\half} - \chi_j^{n+\quart*}) 
             - (P w)_{j-\half}^{n+\quart} (\chibar_{j-\half} - \chi_j^{n+\quart*})
        \right)
      \\[5pt]
  & =  
    & - \frac{\dt}{2\dz} \  
        \frac{1}{2}\left((P w)_{j+\half}^{n+\quart} + (P w)_{j-\half}^{n+\quart}\right) 
        \left(\chibar_{j+\half} - \chibar_{j-\half}\right)
      \\[5pt]
  &   
    & - \frac{\dt}{2\dz} 
        \left( \frac{1}{2}\left(\chibar_{j+\half} - 2\chi_j^{n+\quart*} + \chibar_{j-\half}\right)
               \left( (P w)_{j+\half}^{n+\quart} - (P w)_{j-\half}^{n+\quart} \right)
        \right)
      \nonumber\\[5pt]
  & =  
    & - \frac{\dt}{2\dz} \  
        \frac{1}{2}\left((P w)_{j+\half}^{n+\quart} + (P w)_{j-\half}^{n+\quart}\right) 
        \left(\chibar_{j+\half} - \chibar_{j-\half}\right)
      \\[5pt]
  &   
    & + \frac{\dt}{2} (\dz)^2 \frac{2}{\dt}\left( P_j^{n+\half} - P_j^{n} \right) 
        \frac{1}{2(\dz)^2}\left(\chibar_{j+\half} - 2\chi_j^{n+\quart*} + \chibar_{j-\half}\right)
      \nonumber\\[5pt]
 \label{eq:ConservativeChiUpdate6}
  & =  
    & - \frac{\dt}{2\dz} \  
        \frac{1}{2}\left((P w)_{j+\half}^{n+\quart} + (P w)_{j-\half}^{n+\quart}\right) 
        \left(\chibar_{j+\half} - \chibar_{j-\half}\right)
      \\[5pt]
  &   
    & - \left( P_j^{n+\half} - P_j^{n} \right) \left(\chi_j^{n+\quart*} - \chibar_j^n\right)
      \nonumber\\[5pt]
  &   
    & + \frac{\dt}{16} (\dz)^2 \frac{2}{\dt}\left( P_j^{n+\half} - P_j^{n} \right) 
        \frac{4}{(\dz)^2}\left(\chibar_{j+\half} - 2\chibar_j^n + \chibar_{j-\half}\right)
      \nonumber\\[5pt]
 \label{eq:ConservativeChiUpdate7}
  & = 
    & - \frac{\dt}{2\dz} \  
        \frac{1}{2}\left((P w)_{j+\half}^{n+\quart} + (P w)_{j-\half}^{n+\quart}\right) 
        \left(\chibar_{j+\half} - \chibar_{j-\half}\right)
      \\[5pt]
  &   
    & - \left( P_j^{n+\half} - P_j^{n} \right) \left(\chi_j^{n+\quart*} - \chibar_j^n\right)
      + \frac{\dt(\dz)^2}{16} \left(\pp{P}{t}\right)_{j}^{n+\quart} 
        \left(\ppn{2}{\chibar}{z}\right)_{j}^{n}
      \nonumber
\end{eqnarray}
%
where we have used the abbreviations
%
\begin{equation}
X^{n+\quart*} = \frac{1}{2}\left(X^{n+\half} + X^{n}\right)\,.
\end{equation}
%
While the last term in \eq{eq:ConservativeChiUpdate7} 
is arguably of higher order and may be neglected unless
we want to demonstrate some rigorous bound on $\chi$ or the like, the next
to last term is not negligible. It is the very kind of contribution that 
induces us to carry the $P$-variable along in our scheme even if we solve
the pseudo-incompressible equations despite the fact that in that equation
set, $\partial_t P \equiv 0$: In each split step of our OpSplit procedure, 
that derivative will be nonzero and of order unity in general. 

We understand that our scheme is better than one may think at this point 
if instead of considering only the linearized update for $P\chi$ according
to \eq{eq:LinearizedConservativeChiEqn}, we consider the full nonlinear 
expression
%
\begin{equation}
\pp{P\chi}{t} = - \pp{(P\chi w)}{z}
\end{equation}
%
which turns \eq{eq:ConservativeChiUpdate6} into 
%
\begin{eqnarray}
\frac{2}{\dt} \left(\chi_j^{n+\half} - \chi_j^{n}\right)
  & =  
    & - \frac{1}{2P^{n+\quart*}_j}\left((P w)_{j+\half}^{n+\quart} + (P w)_{j-\half}^{n+\quart}\right) 
        \frac{\chi^{n+\quart}_{j+\half} - \chi^{n+\quart}_{j-\half}}{\dz}      
      \\[5pt]
  &   
    & + \frac{(\dz)^2}{8}  \frac{2}{\dt}\left( P_j^{n+\half} - P_j^{n} \right) 
        \frac{4}{(\dz)^2}\left(\chi^{n+\quart}_{j+\half} - 2\chi^{n+\quart*}_j + \chi^{n+\quart}_{j-\half}\right)
      \nonumber\\
\label{eq:ConservativeChiUpdate8}
  & =  
    & - \frac{1}{2P_j^{n+\quart*}}\left((P w)_{j+\half}^{n+\quart} + (P w)_{j-\half}^{n+\quart}\right) 
        \frac{\chi^{n+\quart}_{j+\half} - \chi^{n+\quart}_{j-\half}}{\dz}      
      \\[5pt]
  &   
    & + \frac{(\dz)^2}{8} \left(\pp{P}{t}\right)_{j}^{n+\quart} 
        \left(\ppn{2}{\chi}{z}\right)_{j}^{n+\quart} + \littleoh{(\dz)^2}\,.
      \nonumber
\end{eqnarray}
%
For a second-order MUSCL-scheme we can do even better. In that case, reconstruction
at $t^{n+\quart}$ is piecewise linear, $\chi^{n+\quart*}_j = \chi^{n+\quart}_j + \bigoh{(\dt)^2}$, 
and the edge values $\chi^{n+\quart}_{j+\half}$ are determined on an upwind basis. 

Suppose, for case 1, that $\chi$ is smooth and does not have an extremum, and that 
$(P w)_{j\pm\half}^{n+\quart} > 0$, then
%
\begin{eqnarray}
\chi^{n+\quart}_{j+\half} - 2\chi^{n+\quart*}_j + \chi^{n+\quart}_{j-\half}
  & =
    & \chi^{n+\quart}_{j,+} - 2\chi^{n+\quart*}_j + \chi^{n+\quart}_{j-1,+}
      \\
  & =
    & \chi^{n+\quart}_{j,+} - 2\chi^{n+\quart}_j + \chi^{n+\quart}_{j,-}
      \\
  & 
    & + \left(\chi^{n+\quart}_{j-1,+} - \chi^{n+\quart}_{j,-}\right)
      - 2 \left(\chi^{n+\quart*}_{j} - \chi^{n+\quart}_{j}\right)
      \\
  & =
    & \left(\chi^{n+\quart}_{j-1,+} - \chi^{n+\quart}_{j,-}\right) + \bigoh{(\dt)^2}
      \\
  & =
    & \left(\chi^{n+\quart}_{j-1} + \frac{\dz}{2} S_{j-1} 
          - \chi^{n+\quart}_{j} + \frac{\dz}{2} S_{j} \right) + \bigoh{(\dt)^2}
      \\
  & =
    & \dz \left(\frac{1}{2} \left(S_{j-1} +  S_{j}\right) 
               - \frac{\chi^{n+\quart}_{j} - \chi^{n+\quart}_{j-1}}{\dz}
          \right) + \bigoh{(\dt)^2}
      \\
  & = 
    & \bigoh{(\dt)^2 + (\dz)^2}
\end{eqnarray}
%

Issues to be addressed: This latter result is not ``better'', as we have
already assumed in the previous estimate that this expression is basically
$(\dz)^2(\partial^2 \chi / \partial z^2)$. What we do have to worry about
are extrema of $\chi$, where the slope reconstruction deteriorates and we
only have a spacial first-order accurate reconstruction, and about the
$\bigoh{(\dt)^2}$ term. \\
That term can deteriorate for large $\dt$, which is the current focus! 
Checking where it comes from, we have to make sure that both 
$\chi_{j,\pm}^{n+\quart}$ (from the reconstruction) and 
$\chi_{j,\pm}^{n+\quart*}$ (from the time update within cell $j$) are
controlled as $\dt \to \infty$ as displayed above in working out the 
explicit part of the semi-implicit time integrator. 

Let us, for the moment, not worry about the second derivative term that
is an unwanted left-over imprint from the time changes of $P$, but rather
focus on the dominant part of the $\chi$-update in 
\eq{eq:ConservativeChiUpdate8}.  
%
\begin{eqnarray}
\frac{2}{\dt} \left(\chi_j^{n+\half} - \chi_j^{n}\right)
  & =  
    & - \frac{1}{2P_j^{n+\quart*}}\left((P w)_{j+\half}^{n+\quart} + (P w)_{j-\half}^{n+\quart}\right) 
        \frac{\chi^{n+\quart}_{j+\half} - \chi^{n+\quart}_{j-\half}}{\dz}      
      + \littleoh{(\dz)^2}\,.
\end{eqnarray}
%
Following the MUSCL strategy translated into the implicit trapezoidal 
context, we insert a discretization of
%
\begin{eqnarray}
w_t 
  & = 
    & - \frac{\theta}{\Gamma}\left(\pi_z^n + \Gamma g \chi\right)
      \\
\chi_t 
  & = 
    & - w \frac{d\chibar}{dz}
\end{eqnarray}
%
over a half time step and then utilize $w^{n+\quart} = \half \left(w^{n+\quart} + w^n\right)$
to extract the half of a half time step level vertical velocities. According to the 
previous considerations in \eq{eq:deltaw3}, this yields
%
\begin{eqnarray}
\label{eq:deltaw4}
w^{n+\half} 
  & = 
    & \frac{1 - \left(\frac{N\dt}{4}\right)^2}{1 + \left(\frac{N\dt}{4}\right)^2}\, w^n
-\frac{\dt/2}{1 + \left(\frac{N\dt}{4}\right)^2} 
\left( \frac{\theta}{\Gamma}\left(\pi^n_z + \Gamma g \chi^n\right)\right)
\\
w^{n+\quart} = \frac{1}{2}\left(w^{n+\half} + w^{n}\right)
  & = 
    & \frac{1}{1 + \left(\frac{N\dt}{4}\right)^2}\,
      \left( w^n -\frac{\dt}{4} g\theta \left(\chi^n + \frac{\pi^n_z}{\Gamma g}\right)
      \right)
\\
  & \equiv 
    & \frac{1}{1 + \left(\frac{N\dt}{4}\right)^2}\,
      \left( w^n -\frac{\dt}{4} g\theta\,\chitilde^n
      \right)
\end{eqnarray}
%
With this we find, neglecting the error term for the moment, and with an 
obvious abbreviation for the vertical $\chi$ derivative,
%
\begin{eqnarray}
P_j^{n+\quart*} \left(\chitilde_j^{n+\half} - \chitilde_j^{n}\right)
  & =  
    & - \frac{\dt}{4}\left((P w)_{j+\half}^{n+\quart} + (P w)_{j-\half}^{n+\quart}\right) 
        \left.\pp{\chi}{z}\right|_j^{n+\quart}     
      \\ 
  & =  
    & - \frac{\dt}{4}
        \left(
        \frac{P_{j+\half}^{n+\quart} w_{j+\half}^{n}}{1 + \left(\frac{N_{j+\half}\dt}{4}\right)^2} 
      + \frac{P_{j-\half}^{n+\quart} w_{j-\half}^{n}}{1 + \left(\frac{N_{j-\half}\dt}{4}\right)^2}
        \right) 
        \left.\pp{\chi}{z}\right|_j^{n+\quart}     
      \\ 
  &
    & - \left(\frac{N^*_j \dt}{4}\right)^2  
       \left(
        \frac{P_{j+\half}^{n+\quart} \chitilde_{j+\half}^{n}}{1 + \left(\frac{N_{j+\half}\dt}{4}\right)^2} 
      + \frac{P_{j-\half}^{n+\quart} \chitilde_{j-\half}^{n}}{1 + \left(\frac{N_{j-\half}\dt}{4}\right)^2}
        \right)  
\end{eqnarray}
%
where
%
\begin{equation}
\left(N_{j*}\right)^2 = - g\theta\left.\pp{\chi}{z}\right|_j^{n+\quart}
\end{equation}
%
and where, on the left hand side, we have replace the $\chi$-difference by
the $\chitilde$-difference by adding a zero.

Thus, what we obtain is
%
\begin{eqnarray}
\chitilde_j^{n+\half} 
  & =  
    & \chitilde_j^{n} 
      - \frac{\left(\frac{N^*_j \dt}{4}\right)^2}{P_j^{n+\quart*}}  
        \left(
        \frac{P_{j+\half}^{n+\quart} \chitilde_{j+\half}^{n}}{1 + \left(\frac{N_{j+\half}\dt}{4}\right)^2} 
      + \frac{P_{j-\half}^{n+\quart} \chitilde_{j-\half}^{n}}{1 + \left(\frac{N_{j-\half}\dt}{4}\right)^2}
        \right)  
      \\ 
  &
    & - \frac{\dt}{2} \frac{1}{2 P_j^{n+\quart*} }
        \left(
        \frac{P_{j+\half}^{n+\quart} w_{j+\half}^{n}}{1 + \left(\frac{N_{j+\half}\dt}{4}\right)^2} 
      + \frac{P_{j-\half}^{n+\quart} w_{j-\half}^{n}}{1 + \left(\frac{N_{j-\half}\dt}{4}\right)^2}
        \right) 
        \left.\pp{\chi}{z}\right|_j^{n+\quart}     
\end{eqnarray}
%
The result does look like a complex variant of the explicit part of the
implicit trapezoidal rule, although with weirdly weighted old time
level data entering. 

Of interest is the limit behavior as $\dt\to\infty$ with everything
else fixed. What we obtain is
%
\begin{equation}
\chitilde_j^{n+\half} \to  \chitilde_j^{n} - 2 \chitilde_j^{n*}
\end{equation}
%
where
%
\begin{equation}
\chitilde_j^{n*} 
= \frac{1}{2}\frac{N^2_{j*}}{P_j^{n+\quart*}}
  \left(\frac{ P_{j+\half}^{n+\quart}}{N^2_{j+\half}}\chitilde^n_{j+\half}
      + \frac{P_{j-\half}^{n+\quart}}{N^2_{j-\half}}\chitilde^n_{j-\half}
  \right)\, .
\end{equation}
%

% ===========================================================================

\subsubsection{Oscillator with slow forcing over a two-step cycle?}
\label{sssec:SlowlyForcedOscillatorTwoStepCycle}

Here we consider 
%
\begin{eqnarray}
y^{n+1} = y^n + \dt v^{n+\half}
\end{eqnarray}
%

\newpage

% ===========================================================================
% ===========================================================================
% ===========================================================================

\section{Todos}

\begin{itemize}

\item advect only perturbations of $1/\theta$
  \begin{itemize}
  \item modify density flux accordingly by superimposing reconstructed
        theta-perturbations and background values at cell faces, or
  \item by advecting theta-perturbations as a separate scalar and then
        recomputing full density after the full advection cycle.
  \item[\red{\textbullet}] \red{Tried this, it doesn't improve things, and is
       also somewhat against our philosophy. So, let's forget about this.}
  \end{itemize}

\item Apply explicit part of first-order backward Euler of the pressure-gravity
      combo to $P$-fluxes, but that of the implicit trapezoidal rule to the 
      cell-centered momenta.
  \begin{itemize}
  \item try just using the current implicit variant on both forward steps -- \dgreen{looks good}
  \item fluxes -- \dgreen{produces controlled updates for very large time steps}
  \item cell-centered momenta -- \blue{partial success: vertical velocity gathers 
        oscillatory but controlled updates, yet the buoyancy does not undergo the
        flip-flop in time that would be expected under the implicit trapezoidal rule
        for large time steps. Reason: advection of background theta is done with the
        first-order updated $P$-fluxes. I have to pull this part out.}
  \item pull background-$\theta$ out of density flux and place it back in a source-term
        like expression. -- \red{This is not appropriate because then I begin to handle
        the density flux not consistently with the fluxes of other density-weighted 
        variables. That should be avoided.}
  \item What about going back to separately computing $\tilde S$, i.e., fluctuations of
        $1/\theta$ only for buoyancy computations and resynchronizing after each step?
  \item[\red{\textbullet}] \red{Meanwile tested applying backward Euler half time step
       for the $P$-fluxes in the first OpSplit cycle and forward Euler in the second.
       This formally should yield second order in the end as it would be the equivalent
       of the implicit midpoint rule, and it does give the most favorable results so far.}
  \end{itemize}

\item 2016.12.01: I seem to be missing some part of the $\theta$-update in calculating
      buoyancy for the final momentum update. Options:
  \begin{enumerate}
  \item Add a final $\theta$-buoyancy contribution when the second projection is done.\\
        \red{I tried this: It does improve the asymmetries in the advected gravity wave
        test case, but it introduces unfavorable behavior in the non-advected case. 
        For rather large time steps, checkerboard modes kick in. I think that the 
        asymmetry arising with mean advection has to do with how I link the advection
        split steps (especially the horizontal ones) to the first projection. \\
        Yet, I still have a suspicion that we cannot keep the implicit gravity discretization
        out of the second projection entirely. So, this is where I stand as of August 18, 2017.}
  \item Memorize all pressure-gradient / gravity terms over the predictor cycle and
        recalculate them after the first projection based on the old and new time level
        $\theta$s and on the nodal pressure. This would effectively provide a true
        trapezoidal rule implementation of these terms albeit with a separate update,
        done in the predictor and first projection, for $\theta$.
  \item We get an interesting variant of this approach when we go back to the nodal-pressure
        only version of the flux divergence control. KEEP THIS IN MIND!
  \item Run scheme as is right now, but flip the $x$- and $y$-sweeps in the predictor. 
        Then we get buoyancy evaluated at the beginning and true end of the $x$-sweep
        of advection. In the end, this is a switch from a midpoint to a trapezoidal 
        discretization. Why should this make a difference? Unclear. Also, we would have
        to implement the decomposition of one semi-implicit predictor into two steps, 
        which is somewhat of an arbitrary procedure.
  \end{enumerate}
 
\item 2016.12.01: The errors in the current version appear to be too large. Before I 
      try modifications that seem all to be variants of a second-order accurate 
      scheme, should I not first hunt for a bug that destroys second order in the first
      place?

\end{itemize}

In \eq{eq:deltaw2} we use the abbreviation $\Gamma g \chi^* = - \pi^{\rm hy}_z$
to obtain
%
\begin{eqnarray}
\label{eq:deltaw3a}
\delta w 
  & = 
    & \frac{1}{1 + \Nscsq} 
      \left(- \tau \frac{\theta}{\Gamma}\left(\pi^*_z - \pi^{\rm hy}_z\right)
            - w^* \Nscsq
      \right)
      - \tau \frac{\theta/\Gamma}{1 + \Nscsq} \delta\pi_z 
      \\
\delta w 
  & = 
    & \delta^{\rm expl} w + \delta^{\rm \pi} w
      \\
\delta \chi
  & = 
    & - \tau\, (w^* + \delta^{\rm expl} w) \frac{d\overline{\chi}}{dz} 
      + \frac{1}{\Gamma g}\frac{1}{1 + \Nscsq} (\tau)^2  g\theta\frac{d\overline{\chi}}{dz} \delta\pi_z 
      \\
\delta \chi
  & = 
    & - \tau\, (w^* + \delta^{\rm expl} w) \frac{d\overline{\chi}}{dz} 
      + \frac{1}{\Gamma g}\frac{\Nscsq}{1 + \Nscsq}\delta\pi_z 
\end{eqnarray}
%
It is clear how to implement this in a split fashion outside of the advection
routines. It needs to be implemented also, however, in the flux calculations, 
and it is less clear how that could be achieved. 

\newpage

% ===========================================================================
% ===========================================================================
% ===========================================================================

\section{Blended hydrostatic--non-hydrostatic pseudo-incompressible model}
\label{sec:HydroBlendPsinc}

% ===========================================================================
% ===========================================================================

\subsection{EULAG-type time integration scheme}
\label{ssec:EULAGTimeIntegrator}

The scheme as of Spring 2018 adopts essentially the overall time integration 
strategy developed by Piotr Smolarkiewicz for his EULAG model. The 
pseudo-incompressible model may be written as 
%
\begin{subequations}\label{eq:PsincI}
\begin{align}
\label{eq:PsincMass}
\dss \rho_t + \nabla_\parallel\cdot(\rho \vu) + (\rho w)_z
  & = \dss 0
      \\[5pt]
\label{eq:PsincHorMom}
\dss (\rho\vu)_t + \nabla_\parallel\cdot(\rho \vu\circ\vu) + (\rho w \vu)_z 
+ \frac{P}{\Gamma}\nabla_\parallel \pi
  & = \dss 0
      \\[5pt]
\label{eq:PsincVerMom}
\dss (\rho w)_t + \nabla_\parallel\cdot(\rho \vu w) + (\rho w^2)_z 
+ \frac{P}{\Gamma} \pi_z
  & = \dss - \rho g
      \\[5pt]
\label{eq:PsincConstraint}
\dss P_t + \nabla_\parallel\cdot(P\vu) + (Pw)_z
  & = \dss 0\,.
\end{align}
\end{subequations}
%
Here $\pi$ is the Exner pressure, $P = \pi^{\frac{1}{\gamma-1}}$,
and the other symbols should be clear from the con- and previous text.
Given \eq{eq:PsincConstraint}, and defining 
%
\begin{equation}
\rho = P\chi \equiv P/\Theta\, ,
\end{equation}
%
The mass transport equation may be rewritten as a transport equation for
$\chi = 1/\Theta$, 
%
\begin{equation}\label{eq:chiI}
\rho_t + \nabla_\parallel\cdot(\rho \vu) + (\rho w)_z = 
(P\chi)_t + \nabla_\parallel\cdot(P\chi \vu) + (P\chi w)_z = 0\,,
\end{equation}
%
or equivalently, for differentiable solutions, as a transport equation

%
\begin{equation}\label{eq:chiII}
\chi_t + \vu\cdot\nabla_\parallel \chi + w \chi_z = 0
\end{equation}
%
which shows that $\chi = 1/\Theta$ is purely advected. 
The equations in \eq{eq:PsincI} are then recast, using \eq{eq:chiI} and
\eq{eq:chiII} and splitting $\chi(t,\vx,z) = \chitilde(t,\vx,z) + \chibar(z)$, 
and introducing the blending parameters $\rbeta$ and $\balpha$ for 
hydrostasy and pseudo-incompressibility, respectively,  
%
\begin{equation}\label{eq:PsincII}
\begin{array}{rcrcl}
\dss (P\chitilde)_t 
  & + 
    & \dss \nabla_\parallel\cdot(P\vu\, \chitilde) + (Pw\, \chitilde)_z
      & = 
        & \dss -\left[\nabla_\parallel\cdot(P\vu\, \chibar) + (Pw\, \chibar)_z\right]
          \\[10pt]
\dss (P\vu)_t 
  & + 
    & \dss \nabla_\parallel\cdot(P\vu\circ\vu) + (Pw\, \vu)_z 
      & = 
        & \dss - \frac{P\Theta}{\Gamma}\nabla_\parallel \pi
          \\[10pt]
\dss \rbeta\, \Bigl[(P w)_t 
  & + 
    & \dss \nabla_\parallel\cdot(P \vu\, w) + (Pw\, w)_z\Bigr] 
      & = 
        & \dss - \left(\frac{P\Theta}{\Gamma} \pi_z + P g\right)
          \\[15pt]
\balpha\, P_t
  &  
    & \dss \dss \nabla_\parallel\cdot(P\vu)  + (Pw)_z
      & = 
        & \dss 0\,.
\end{array}
\end{equation}
%
Here the second column of terms on the left represents advection of 
$\Psi = (\chi, \vu, w)$ by the advecting fluxes $P\vv \equiv (P\vu, Pw)$, 
which are divergence-controlled, and the right hand side column presents 
the forces acting to change the momentum field. These equations are 
written in compact form as
%
\begin{equation}
(P\Psi)_t + \mathcal{A}(\Psi; P\vv) = R(\Psi)\, ,
\end{equation}
%
and with this notation the EULAG-type time integration scheme for an update
of the solution $\Psi$ from time $t^n$ to time $t^{n+1}$ reads 
%
\begin{equation}\label{eq:EULAGTimeIntegrator}
\begin{array}{rcl}
\dss \Psi^{*} 
  & = 
    & \dss \Psi^{*} + \frac{\Delta t}{2} R(\Psi^n)
      \\
\dss \Psi^{**} 
  & = 
    & \dss \mathcal{A}^{\Delta t}\left(\Psi^*; (P\vv)^{n+1/2}\right)
      \\
\dss \Psi^{n+1} 
  & = 
    & \dss \Psi^{**} + \frac{\Delta t}{2} R(\Psi^{n+1})\,.
      \\
\end{array}
\end{equation}
%
Here $\mathcal{A}^{\Delta t}\left(\Psi; (P\vv)\right)$ denotes a
second-order accurate explicit approximation of the \emph{linear} 
advection of $\Psi$ by the \emph{given} advecting fluxes $(P\vv)$ 
over a time increment $\Delta t$.

The advecting fluxes $(P\vv)^{n+1/2}$ used in the advection step 
are at least first order accurate approximations of the $P\vv$ field
at the half time level $t^{n+1/2} = (t^n + t^{n+1})/2$ that can be
determined in various ways. The default option in EULAG is extrapolation
from the last two completed time levels, the option adopted here and
in an alternative variant of EULAG is to carry out an intermediate 
(possibly lower-order accurate) time step to the half time level to 
evolve them from $(P\vv)^{n}$ without the need to recur to the $t^{n-1}$
level solution. 

With these preliminaries, the scheme in \eq{eq:EULAGTimeIntegrator} is
best interpreted as the application of the implicit trapezoidal rule
for time integrating the source terms $R(\Psi)$, anchored at the  
beginning and end points of Lagrangian trajectories traced between the
pertinent time levels. A forward Euler step is followed by a backward
Euler step, with the advection step interleaved so as to represent
the advancement of the Lagrangian paths.  

An observation that should be important in analyzing the (remarkable)
stability features of EULAG, which seems to be inherited by the current 
implementation, is that for given $(P\vv)$ the advection step is \emph{linear}. 
As a consequence, even if for rather large time steps the velocities 
computed in the forward Euler step end up far away from the solutions 
at time levels $n$ or $n+1$, these excursions do (i) not affect the 
advecting fluxes $(P\vv)^{n+1/2}$, and they can (ii) properly be 
post-corrected and pulled back to the reasonable range by the backward 
Euler step. 

The half time step predictor for the advecting fluxes needs to be first
order accurate only to preserve second order accuracy of the entire 
scheme, and thus it is legitimate to use the following simplified 
version of \eq{eq:EULAGTimeIntegrator} for this purpose:
%
\begin{equation}\label{eq:EULAGHalfTimePredictor}
\begin{array}{rcl}
\dss \Psi^{\#} 
  & = 
    & \dss \mathcal{A}^{\frac{\Delta t}{2}}\left(\Psi^{n}; (P\vv)^{n}\right)
      \\
\dss \Psi^{n+1/2} 
  & = 
    & \dss \Psi^{\#} + \frac{\Delta t}{2} R(\Psi^{n+1/2})\,,
      \\
\end{array}
\end{equation}
%
which corresponds to a simple first order splitting scheme.

The large time step stability of the scheme is then determined by 
the final implicit half time steps in \eq{eq:EULAGTimeIntegrator}
and \eq{eq:EULAGHalfTimePredictor}, which are conceptually identical, 
except that \eq{eq:EULAGTimeIntegrator} leads to divergence-controlled
cell-centered velocities, whereas \eq{eq:EULAGTimeIntegrator} leads to
divergence-controlled advective fluxes at the cell faces.

% ===========================================================================
% ===========================================================================

\subsection{Blended backward Euler step}
\label{ssec:BlendedBackwardEuler}

The linearized backward Euler discretization for the equation 
$\Psi_t = R(\Psi)$ between two time levels $n, n+1$ with time step
$\Delta t$ reads
%
\begin{subequations}\label{eq:BackwardEulerStep}
\begin{align}
\label{eq:BWETheta}
\dss \frac{(P\Thetatilde)^{n+1}-(P\Thetatilde)^{n}}{\Delta t} 
  & = \dss - (Pw)^{n+1} \frac{d\Thetabar}{dz}
      \\[5pt]
\label{eq:BWEHorMom}
\dss \frac{(P\vu)^{n+1} - (P\vu)^{n}}{\Delta t} 
  & = \dss - \frac{(P\Theta)^*}{\Gamma}\nabla_\parallel \pitilde^{n+1}
      \\[5pt]
\label{eq:BWEVerMom}
\dss \rbeta\ \frac{(Pw)^{n+1} - (Pw)^{n}}{\Delta t} 
  & = \dss - \left(\frac{(P\Theta)^*}{\Gamma} \pitilde^{n+1}_z - \frac{(P\Thetatilde)^{n+1} g}{\Thetabar}\right)
      \\[5pt]
\label{eq:BWEConstraint}
\dss \balpha\, \frac{P^{n+1} - P^n}{\Delta t}
+ \nabla_\parallel\cdot(P\vu)^{n+1} + (Pw)^{n+1}_z
  & = \dss 0\,.
\end{align}
\end{subequations}
%
Here we have used the fact that if $\chi$ is an advected quantity, then so 
is $\Theta = 1/\chi$, and we have subtracted a background hydrostatic 
distribution from the Exner pressure to work with the perturbation  
$\pitilde(t,\vx,z) = \pi(t,\vx,z) - \pibar(z)$. The background Exner
pressure satisfies $\pibar_z = -\Gamma g/\Thetabar$; $\pibar(0) = 1$.
 
 
% ===========================================================================

\subsubsection{Hydrostatic--non-hydrostatic ``$\rbeta$''-blend for the psinc model ($\balpha = 0$)}
\label{sssec:HydroBlend}

Setting $\balpha = 0$ for now, 
we aim to derive a Poisson-type equation for the Exner pressure perturbation, 
$\pitilde^{n+1}$. To this end, we first reformulate \eq{eq:BWETheta} as
%
\begin{equation}
\frac{g(P\Thetatilde)^{n+1}}{\Thetabar} 
= \frac{g(P\Thetatilde)^{n}}{\Thetabar} - \Delta t\, (Pw)^{n+1} N^2\,,
\end{equation}
%
where 
%
\begin{equation}
N^2 = \frac{g}{\Thetabar}\frac{d\Thetabar}{dz}
\end{equation}
%
is the Background Brunt-V\"ais\"al\"a frequency. Next we insert this 
result in \eq{eq:BWEVerMom} and solve for 
%
\begin{equation}\label{eq:BWEWPrediction}
(Pw)^{n+1} = 
\frac{\rbeta (Pw)^{n} + \Delta t \frac{g(P\Thetatilde)^{n}}{\Thetabar}}%
{\rbeta + \Delta t^2 N^2}
- \Delta t\frac{(P\Theta)^*}{\Gamma (\rbeta + \Delta t^2 N^2)} \pitilde^{n+1}_z\,.
\end{equation}
%
Similarly, the horizontal momentum equation \eq{eq:BWEHorMom} yields
%
\begin{equation}
(P\vu)^{n+1} 
= (P\vu)^{n} - \Delta t \, \frac{(P\Theta)^*}{\Gamma} \nabla_\parallel\pitilde^{n+1}
\end{equation}
%
These last two expressions are next inserted in the divergence constraint 
\eq{eq:BWEConstraint} to obtain the pressure perturbation equation
%
\begin{equation}
\nabla_\parallel\cdot\left(\frac{(P\Theta)^*}{\Gamma}\nabla_\parallel\pitilde^{n+1}\right)
+ \left(\frac{(P\Theta)^*}{\Gamma(\rbeta + \Delta t^2 N^2)} \pitilde^{n+1}_z\right)_z
= \frac{1}{\Delta t}\left(\nabla_\parallel\cdot(P\vu)^n + (Pw)^{*,\rbeta}_z\right)\,,
\end{equation}
%
where $(Pw)^{*,\rbeta}$ is the first term on the right of 
\eq{eq:BWEWPrediction}.

As long as $\Delta t^2 N^2 > 0$, this equation allows us to continuously 
blend from the non-hydrostatic case with $\rbeta = 1$ to the hydrostatic 
one with $\rbeta = 0$ without any explicit ``{\tt if}''-decision switch.
Result of the reference code for the planetary scale IGW test come out
with virtually identical solutions. Also, for relatively large values of
$\Delta t^2 N^2 > 0$, the condition of the elliptic problem is not affected
by the blending parameter significantly, so that computational efficiency
remains uniform throughout.

% ===========================================================================

\subsubsection{Compressible--psinc ``$\balpha$''-blend for 
the non-hydrostatic case ($\rbeta = 1$)}
\label{sssec:PsincBlend}

To be done.

% ===========================================================================

\subsubsection{Full compressible--psinc / hydro--non-hydro ``$\rbeta, \balpha$''-blend}
\label{sssec:HydroPsincBlend}

To be done.

\newpage

% ===========================================================================
% ===========================================================================
% ===========================================================================

\section{Preconditioning}
\label{sec:Precon}

\red{So far, this is a side issue. It looks as if the code converges very
rapidly when all the components fit well together, even with vanilla
BiCGStab. We can get back to this when we have the basic scheme running
well. }

% ===========================================================================
% ===========================================================================

\subsection{Introductory thoughts}
\label{ssec:PreconIntro}

Let me recall:  Given $A \in \reals^{n\times n}$, $b \in \reals^n$ find 
$x \in \reals^n$ such that
%
\begin{equation}
A x = b
\end{equation}
%
Preconditioning from the left means multiplying by a matrix 
$B^{-1} \in \reals^{n\times n}$ and considering the problem
%
\begin{equation}
(B^{-1}A) x = B^{-1}b
\end{equation}
%
with the hope that $B^{-1}A$ is better conditioned than $A$. 

Using this strategy, a solver will remain unchanged except for 
%
\begin{enumerate}
\item Computation of the rescaled right hand side $b \to B^{-1}b$ and
\item Extending the call to $Ax$ by the composition of calls $B^{-1}\circ A\, x$. 
\end{enumerate}
%
An interesting question besides this more or less trivial operation 
concerns the computation of the stopping criterion for an iterative
scheme. In some sense, lacking the exact solution, one will want to
use information on the residual to assess how close one has come to 
the solution of the problem. 

Thus, there are two alternatives, 
%
\begin{equation}
\parallel b - Ax^n \parallel < \eps
\qquad\text{or}\qquad
\parallel B^{-1}b - (B^{-1}A)x^n \parallel < \eps \, .
\end{equation}
%
What are the implications? 

To get an intuition, let's take the standard case in meteorology where
we need to compute a solution in a deep atmosphere, across which the
density falls off by a factor $10^{-4}$ or an even more extreme one. 
In the present implementation that solves the 
$P \equiv \overline{\rho\theta}$ equation implicitly and uses an 
Exner type pressure variable, $x \equiv \delta\pi$, the matrix $A$ is 
dominated by the discrete approximation of
%
\begin{equation}
\nabla\cdot\left(\frac{P\theta}{\Gamma} \nabla \delta\pi\right)
\sim 
\nabla\cdot\left(\rho\theta^2 \nabla \delta\pi\right)
\end{equation}
%
and the right hand side is 
%
\begin{equation}
b = \nabla\cdot (P\vv)^*\,,
\end{equation}
%
where the $(\ )^*$ superscript denotes the predicted values before the
implicit divergence controlling step. The divergence constraint 
in the pseudo-incompressible limit reads $\nabla\cdot (P\vv) = 0$, but
the quantity of real interest is the velocity field. Therefore, we 
would want the factor of $P$, which has a dynamic range comparable to the
density, to be scaled out of the div-control. This means we would want 
to control
%
\begin{equation}
{\rm diag}(P)^{-1} \left(b - Ax\right) < \ \parallel {\rm diag}(P)^{-1}\parallel\ \eps\, .
\end{equation}
%
Since division by the diagonal (or the main culprit for extreme dynamic
range in the diagonal) corresponds simply to diagonal preconditioning, 
the answer to the question raised above should be:  Control the residual
of the preconditioned problem,
%
\begin{equation}
\parallel B^{-1} (b - A) x^n \parallel\  < \ 
\parallel B^{-1}\parallel\ \eps \, ,
\end{equation}
%
and not the raw residuum.

% ===========================================================================
% ===========================================================================

\subsection{Preconditioning w.r.t.\ vertical columns}
\label{ssec:PreconVerticalColumn}

% ===========================================================================

\subsubsection{Columnwise preconditioning for the first projection}
\label{ssec:PreconVerticalColumnFirstProjection}

The pressure stencil for the first projection can be read as follows:
%
\begin{equation}
\begin{split}
\left(\Lopfirst \pi\right)_{i,j} 
  & = \frac{h^x_{i+\half,j}\left(\pibar^y_{i+1,j} - \pibar^y_{i,j}\right)
          - h^x_{i-\half,j}\left(\pibar^y_{i,j} - \pibar^y_{i-1,j}\right)}{dx^2}
    \\[10pt]
  & + \frac{h^y_{i,j+\half}\left(\pibar^x_{i,j+1} - \pibar^x_{i,j}\right)
          - h^y_{i,j-\half}\left(\pibar^x_{i,j} - \pibar^x_{i,j-1}\right)}{dy^2}
\end{split}
\end{equation}
%
where 
%
\begin{equation}
\begin{split}
\pibar^x_{i,j}
  & = \frac{\alpha}{2} \pi_{i-1,j} + (1-\alpha)\, \pi_{i,j} + \frac{\alpha}{2} \pi_{i+1,j} 
    \\[10pt]
\pibar^y_{i,j}
  & = \frac{\alpha}{2} \pi_{i,j-1} + (1-\alpha)\, \pi_{i,j} + \frac{\alpha}{2} \pi_{i,j+1} 
\end{split}
\end{equation}
%
The dominant vertical derivative part of the operator thus reads
%
\begin{equation}
\begin{split}
\left(\Lopfirst \pi\right)_{j} 
  & = \frac{h^y_{i,j+\half}\pibar^x_{i,j+1} 
          - \left(h^y_{i,j+\half} + h^y_{i,j-\half}\right) \pibar^x_{i,j}
          + h^y_{i,j-\half}\pibar^x_{i,j-1}}{dy^2}
\end{split}
\end{equation}
%
where 
%
\begin{equation}
\begin{split}
\pibar^x_{i,j}
  & = \frac{\alpha}{2} \pi_{i-1,j} + (1-\alpha)\, \pi_{i,j} + \frac{\alpha}{2} \pi_{i+1,j} \,.
\end{split}
\end{equation}
%
To solve the equation
%
\begin{equation}
\left(\Lopfirst \pi\right)_{j} = r_j
\end{equation}
%
we can obviously first solve for $\pibar$ using the Thomas Algorithm
for the $j$-direction, and then for $\pi$ by inverting the $x$-averaging -- 
again using the Thomas Algorithm. 


\newpage

% ===========================================================================
% ===========================================================================
% ===========================================================================

\section{Hydrostatic initialization}
\label{sec:HydroInit}

The semi-implicit part of the time integration scheme that we intend to 
use is either implicit trapezoidal or implicit midpoint. In both cases
we have an explicit contribution (Euler forward) followed or preceded by
an implicit one (Euler backward). Especially with my implementation up
to Jan 2018, the Euler forward is first, and this implies potentially 
very large deviations from the balanced state in the course of the first
few time steps. This is indeed observed. Now, due to the advective 
nonlinearity, these large deviations can leave a heavy imprint on the
later solution, and this must be avoided. 

Here I describe a pressure initialization by the hydrostatic model. This
is a preliminary exercise also on our way towards constructing a blended
hydrostatic/nonhydrostatic solver. 

The linearized hydrostatic pseudo-incompressible equations for a vertical
slice read
%
\begin{subequations}
\begin{eqnarray}
\label{eq:HorizontalMomentum}
(P\vu)_t + \frac{P\theta}{\Gamma} \pi_x
  & = 
    & 0
      \\
\label{eq:Hydrostatics}
\frac{\theta}{\Gamma}\pi_z
  & = 
    & -  g 
      \\
\theta_t
  & = 
    & - w \frac{d\overline{\Theta}}{dz} 
      \\[5pt]
\label{eq:DivConstraint}
(P\vu)_x + (PW)_z
  & =
    & 0
\end{eqnarray}
\end{subequations}
%
Integrate \eq{eq:DivConstraint} from $z=0$ to $z=h$ and then from 
$0$ to $x$, use the rigid lid conditions on $w$, take the time derivative,
let $\pi = \pi_0(t,x) + \widetilde\pi(t,x,z)$, where
%
\begin{equation}
\widetilde\pi(t,x,z) = - \int\limits_0^z \frac{\Gamma g}{\theta(t,x,\zeta)} \, d\zeta
\end{equation}
%
and obtain
%
\begin{equation}
\int\limits_0^h (P\vu)_t\, dz 
= \left\langle (PU)_t \right\rangle(t)
= - \left\langle\frac{P\theta}{\Gamma}\right\rangle \pi^0_x - 
  \left\langle\frac{P\theta}{\Gamma}\widetilde\pi_x \right\rangle \,.
\end{equation}
%
This yields
%
\begin{equation}
\pi^0_x
= - \frac{\left\langle (PU)_t \right\rangle }{\left\langle P\theta/\Gamma\right\rangle} 
  - \frac{\left\langle P\theta\widetilde\pi_x/\Gamma \right\rangle}%
         {\left\langle P\theta/\Gamma\right\rangle}\,.
\end{equation}
%
Integration in $x$ yields the bottom pressure $\pi^0$ when 
$\left\langle (PU)_t \right\rangle$ is adjusted so as to satisfy the 
pressure boundary conditions, e.g., periodic ones.

% ===========================================================================
% ===========================================================================
% ===========================================================================

\section{Newton iteration for the $P$-equation}
\label{sec:NewtonP}

The determination of the advective fluxes $P\vv$ at the half time level involve
a backward Euler discretization of the half time step for
%
\begin{equation}
P_t + \nabla\cdot(P\vv) = 0\,.
\end{equation}
%
Starting from an initial explicit guess that already includes the advection
terms, the semi-implicit discretization is written for the Exner pressure
variable as
%
\begin{equation}\label{eq:SemiImplicitNonlinearP}
\frac{P(\pi^{n+\half}) - P^n}{\Delta t/2}
+ \nabla\cdot
  \left( (P\vv)^{*} - \frac{\Delta t}{2} P\theta\nabla\pi^{n+\half}
  \right) 
  = 0\,.
\end{equation}
%
The function $P(\pi)$ is nonlinear and we intend to solve 
\eq{eq:SemiImplicitNonlinearP} by a Newton iteration.
Let $\nu$ denote the Newton iteration counter and
%
\begin{equation}
\delta\pi^{\nu} = \pi^{n+\half,\nu+1}-\pi^{n+\half,\nu}\,,
\end{equation}
% 
then the linearization of the pressure function according to Newton's method yields
%
\begin{equation}\label{eq:NewtonPrelimI}
\frac{\frac{\partial P}{\partial \pi}\delta\pi^\nu + P(\pi^{n+\half,\nu}) - P^n}{\Delta t/2}
+ \nabla\cdot
  \left( (P\vv)^{*} - \frac{\Delta t}{2} P\theta\nabla\left[\pi^{n+\half,\nu} + \delta\pi^{\nu}\right]
  \right) 
  = 0\,,
\end{equation}
%
where we assume the coefficients $\frac{\partial P}{\partial \pi}$ and
$P\theta$ to be constant as assessed after the first explicit half time
step. Improvements can be developed later as part of the iteration.

Thus, the Newton iteration requires the solution of
%
\begin{equation}\label{eq:NewtonPrelimIa}
-\frac{4}{\Delta t^2}\frac{\partial P}{\partial \pi}\delta\pi^\nu
+ \nabla\cdot
  \left( P\theta\nabla\delta\pi^{\nu}
  \right) 
  = \frac{4}{\Delta t^2} \left(P(\pi^{n+\half,\nu}) - P^n\right)
    + \frac{2}{\Delta t}\nabla\cdot \left( (P\vv)^{n+\half,\nu} \right)\,.
\end{equation}
%
Considering \eq{eq:NewtonPrelimI} with $\nu$ replaced with $\nu-1$ and 
recalling that $\pi^{n+\half,\nu-1} + \delta\pi^{\nu-1} = \pi^{n+\half,\nu}$, 
we have
%
\begin{equation}\label{eq:NewtonPrelimII}
\frac{2}{\Delta t} \nabla\cdot \left( (P\vv)^{n+\half,\nu} \right) = 
- \frac{4}{\Delta t^2}
\left(\frac{\partial P}{\partial \pi}\delta\pi^{\nu-1} + P(\pi^{n+\half,\nu-1}) - P^n\right)\,,
\end{equation}
%
and re-insertion into \eq{eq:NewtonPrelimII} yields
%
\begin{equation}
-\frac{4}{\Delta t^2}\frac{\partial P}{\partial \pi}\delta\pi^\nu
+ \nabla\cdot
  \left( P\theta\nabla\delta\pi^{\nu}
  \right) 
  = \frac{4}{\Delta t^2} 
    \left( \left[P(\pi^{n+\half,\nu}) - P(\pi^{n+\half,\nu-1})\right] 
         - \frac{\partial P}{\partial \pi}\delta\pi^{\nu-1}\right)\,.
\end{equation}
%
and this constitutes the recursion relation for the right hand side of 
subsequent Newton steps.

The start of the iteration consists of the original step in which the
nonlinear $P$-$\pi$-relation is simply linearized, i.e., we let
%
\begin{equation}
\pi^{n+\half,0} = \pi^n
\qquad\text{and}\qquad
P(\pi^{n+\half,0}) - P^n = 0
\end{equation}
%
according to the result of the iteration in the previous time step 
as a first guess, and then \eq{eq:NewtonPrelimI} can be cast into 
the standard form for the semi-implicit, linearized scheme for
$\pi^{n+\half,1} = \pi^{n} + \delta\pi^0$, 
%
\begin{equation}
-\frac{4}{\Delta t^2}
\frac{\partial P}{\partial \pi}\pi^{n+\half,1}
+ \nabla\cdot
  \left(P\theta\nabla\left[\pi^{n+\half,1}\right]
  \right) 
  = -\frac{4}{\Delta t^2} \frac{\partial P}{\partial \pi}\pi^{n}
+ \frac{2}{\Delta t}\nabla\cdot (P\vv)^{*}\,,
\end{equation}
%
and then the iteration starts with
%
\begin{equation}
\delta\pi^0 = \pi^{n+\half,1} - \pi^{n}  \,.
\end{equation}
%
% ===========================================================================
\newpage
\section{Summary of the numerical scheme}

\small

Here we report for comparison the summary of the numerical scheme as in \citep{Benacchio2014} \benacchio{In addition to the numerical differences, your code also works with Exner $\pi$ rather than $p$.}
%
\begin{algorithm}[Compressible scheme]
\ \\
\begin{description}
%
\item[\emph{Predictor}]
%
\begin{enumerate}\item[]
\item[(i)] Compute linearly reconstructed $u\in\{P,\,\vv\}$-values at the interfaces ($x_{i+1/2},\,y_j$) via     
\begin{subequations}
\label{eq: vel_lin_recon}
\begin{align}
\dss
u_L
& =
\dss
u_i+\half\psi(u_i-u_{i-1}, u_{i+1}-u_i)
\\
u_R
& =
\dss
u_{i+1}-\half\psi(u_{i+1}-u_i, u_{i+2}-u_{i+1})
\end{align}
\end{subequations}
%
where $\psi$ is a slope function. In particular, we have:
%
\begin{equation}
\dss
\psi(a, b)=\dfrac{a+b}{2}
\end{equation} 
%
for centred slopes and compute advecting upwind fluxes of $P$ via
\begin{equation}
\label{eq: avg_l_r_flux_P}
\dss
F(P)=F^+(P)+F^-(P)
\end{equation} 
%
where:
%
\begin{equation}
\dss
F^+(P)=P_L\max(\vu, 0),\qquad F^-(P)=P_R\min(\vu, 0) 
\end{equation} 
\item[(ii)] Linearly reconstruct $1/\theta$ and $\vv/\theta$ at the cell interfaces and compute the remaining advecting fluxes via \begin{equation}
\label{eq: avg_l_r_flux_phi}
\dss
F(\phi)=F^+(P)\phi_L+F^-(P)\phi_R
\end{equation} 
where: 
$\phi\in\left\{\theta^{-1}, \vv\theta^{-1}\right\}$. 


For the momentum, interpolate the nodal pressure values to the cell interface centres for the flux computation;
\item[(iii)] Compute new-time-level predicted quantities using:
\begin{subequations}\label{eq: disc_syst_pred}
\begin{alignat}{4}
\dss 
\rho_C^{n+1, *}  
& =  
& \dss 
\rho_C^{n} 
& -   
\dss 
\Delta t 
\left(\widetilde{\nabla}\cdot
\left(P\vv \, \theta^{-1}  
\right)
\right)^{n+\frac{1}{2}, *}_C\, ,
\label{eq: predconteq}
\\
\dss 
\left(\rho\vv\right)_C^{n+1, *}  
& = 
& \dss \
\left(\rho\vv\right)_C^{n}
& -  
\dss 
\Delta t 
\left( \widetilde{\nabla}\cdot
\left(P\vv \circ \vv\theta^{-1} + p^n\mathbb{I}
\right)
\right)^{n+\frac{1}{2}, *}_C
- 
\dss 
\Delta t \, 
g\vk
\left( \rho \right)^{n+\frac{1}{2}, *}_C \, ,
\label{eq: pred_mom_disc}
\\
\dss P^{n+1, *}_C  
& = 
&  \dss 
P_C^{n} 
& -
\dss 
\Delta t 
\left(\widetilde{\nabla}\cdot
\left(P\vv  
\right)
\right)^{n+\frac{1}{2}, *}_C\, . 
\label{eq: pred_enP_disc}
\end{alignat}
\end{subequations}

\end{enumerate}
%

\benacchio{The above was done with Runge-Kutta time integration in \citep{Benacchio2014}. Instead, (one version of) your code has MUSCL-like reconstructions, i.e. eqs. (56) and (57), and time-marching, eq. (55),
in \citep{KleinTCFD2009}. As discussed back in January, this has now changed  and the target scheme is an implicit-then-explicit predictor with two half steps along the lines of sections 1 and 2 in
this document.}

\item[\emph{First Correction}]
%
\begin{enumerate}\item[]
\item[(i)] Solve the Helmholtz equation
%
\begin{equation}
\label{eq: first_corr_helm_eq2_xi1}
\dss 
-\left(\dfrac{\mathcal{C}_H^{n+\half, *}}{\Delta t}\delta p\right)_C
+
\dss
\widetilde{\nabla}\cdot\left[\frac{\Delta t}{2}\theta^{n+\half, *}\nabla\delta p\right]_C
=
\dss
\widetilde{\nabla}\cdot[(P\vv)^{n+\half. *}]_C
\end{equation} 
%
for cell-centred pressure increments;
\item[(ii)] Use the solution to correct advecting fluxes via
\begin{subequations}\label{eq: first_corr_pred_val_adj}
\begin{align}
\dss
&
\rho_C^{n+1}
=
\dss
\rho_C^{n+1, *}
-
\dss
\Delta t\widetilde{\nabla}\cdot\left(\delta P\vv\theta^{-1}\right)_C
\\
\dss 
&
(\rho\vv)_C^{n+1, **}
=
(\rho\vv)_C^{n+1, *}
-
\dss
\Delta t\widetilde{\nabla}\cdot\left(\delta P\vv\circ\vv\theta^{-1}\right)_C\label{eq: first_corr_mom_corr}
\\
\dss
&
P_C^{n+1}
=
\dss
P_C^{n+1, *}
-
\dss
\Delta t\widetilde{\nabla}\cdot\left(\delta P\vv\right)_C.\label{eq: first_corr_en_corr}
\end{align}
\end{subequations}
%
\end{enumerate}
%
\benacchio{In (one version of) your code, the first projection, including implicit gravity treatment, is now ported between the two half steps in the predictor step. The mean potential temperature advection is captured inside
the corrector, while the perturbations are handled in the explicit step (I am not sure I understand this, I think we are talking about equations (78) and (85) in the other document, RKLM$\_$Documentation.pdf).}

\item[\emph{Second Correction}]
%

\begin{enumerate}\item[]
\item[(i)] Solve the Helmholtz equation:
%
\begin{equation}
\label{eq: second_corr_helmholtz}
\dss
\left(-\dfrac{\mathcal{C}^{n+1}_H}{\Delta t}\delta p_\nu\right)_{\overline{C}}
+
\widetilde{\nabla}\cdot\left(\theta\frac{\Delta t}{2}\theta^{n+1}\nabla\delta p_\nu\right)_{\overline{C}}
= \widetilde{\nabla}\cdot\left[\theta(P\vv)^{n+1, **}+(1-\theta)(P\vv)^n\right]_{\overline{C}}
\end{equation}  
%
for node-centred pressure increments;
\benacchio{Presumably this changes to include buoyancy terms along the lines of Chapter 6 \citep{Benacchio2014}, i.e. incorporating the terms containing the mean $\theta$?}

\item[(ii)] Interpolate the solution on cell interface centres and correct momentum flux via 
\begin{equation}
\label{eq: final_mom_upd}
\dss
(\rho\vv)^{n+1}_C 
= 
(\rho\vv)^{n+1,**}_C 
-
\dss
\dfrac{\Delta t}{2}\widetilde{\nabla}\cdot(\delta p\mathbb{I})_C.
\end{equation}
\end{enumerate}
%
\item[\emph{Pressure update}]\ \\ Synchronise the energy and pressure via the equation of state using the binding formula 
%
\begin{equation}
\label{eq: final_press_update_eos}
\dss
p_C^{n+1}
=
\dss
\left(\dfrac{P^{n+1}}{\rho_\textrm{ref} T_\textrm{ref}}\right)^\gamma p_{\textrm{ref}}.
\end{equation} 
%
\benacchio{I understand you have an option for this in your code? If so, is it used in all cases and is it the case that results are better with it than without it as it was for my model?}
\end{description}
\end{algorithm}

% ===========================================================================
% ===========================================================================
% ===========================================================================

\bibliographystyle{FluidMechanics}
\bibliography{Bibliography}

\end{document}