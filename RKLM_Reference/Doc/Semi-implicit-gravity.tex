\documentclass[11pt,a4]{article}

\parindent=0pt
\parskip=3pt

\setlength{\paperheight}{29.5cm}
\setlength{\paperwidth}{21.2cm}

%\setlength{\voffset}{-2.0cm}
\setlength{\headheight}{0cm}
\setlength{\headsep}{0cm}
\setlength{\textheight}{23.7cm}
\setlength{\textwidth}{16cm}
\setlength{\oddsidemargin}{0cm}
\setlength{\evensidemargin}{0cm}
\setlength{\topmargin}{0cm}
\setlength{\topskip}{0cm}

\catcode`\"=\active \let"=\"
\let\3=\ss

\usepackage[latin9]{inputenc}

\usepackage[sort]{natbib}
\bibpunct{(}{)}{,}{a}{}{,}

\usepackage{amsmath}
\usepackage{amssymb}

%\usepackage[squaren]{SIunits}
\usepackage{graphicx}

%----------------------------------------------------------------------------
%---- Farben und Macros zur Textmarkierung ----------------------------------
%----------------------------------------------------------------------------
\usepackage{color}
\definecolor{light}{gray}{0.50}
\definecolor{heavy}{gray}{0.35}
\definecolor{black}{gray}{0.0}
\definecolor{dgreen}{rgb}{0.0,0.7,0}
\definecolor{dred}{rgb}{0.9959,0,0}
\definecolor{green}{rgb}{0.0,0.99599,0.0}
\definecolor{purple}{rgb}{0.6,0.0,0.4}

\newcommand{\red}[1]{\textcolor{dred}{#1}}
\newcommand{\green}[1]{\textcolor{green}{#1}}
\newcommand{\dgreen}[1]{\textcolor{dgreen}{#1}}
\newcommand{\purple}[1]{\textcolor{purple}{#1}}
\newcommand{\blue}[1]{\textcolor{blue}{#1}}
\newcommand{\black}[1]{\textcolor{black}{#1}}
\newcommand{\grey}[1]{\textcolor{heavy}{#1}}
\newcommand{\lightgrey}[1]{\textcolor{light}{#1}}

%----------------------------------------------------------------------------
%---- Kommentare ------------------------------------------------------------
%----------------------------------------------------------------------------

\newcommand{\klein}[1]{\textcolor{blue}{#1}}
\newcounter{kleincommentno}
\setcounter{kleincommentno}{1}
\newcommand{\kleincomment}[1]{{\small\bfseries\textcolor{blue}{ {\small\bfseries${}^{[\arabic{kleincommentno}]}$}}%
\marginpar{\textcolor{blue}{{\small{\bfseries[\arabic{kleincommentno}]}\ \small #1}}\addtocounter{kleincommentno}{1}}}}

\newcommand{\benacchio}[1]{\textcolor{red}{#1}}
\newcounter{benacchiocommentno}
\setcounter{benacchiocommentno}{1}
\newcommand{\benacchiocomment}[1]{{\small\bfseries\textcolor{red}{ {\small\bfseries${}^{[\arabic{benacchiocommentno}]}$}}%
\marginpar{\textcolor{red}{{\small{\bfseries[\arabic{benacchiocommentno}]}\ \small #1}}\addtocounter{benacchiocommentno}{1}}}}

%----------------------------------------------------------------------------
%---- Eigene Macros ---------------------------------------------------------
%----------------------------------------------------------------------------
\newcommand{\hsc}{h_{\rm sc}}
\newcommand{\order}[1]{^{(#1)}}
\let\dss=\displaystyle

\renewcommand{\vector}[1]{\relax\ifmmode\mathchoice
{\mbox{\boldmath$\displaystyle#1$}}
{\mbox{\boldmath$\displaystyle#1$}}
{\mbox{\boldmath$\scriptstyle#1$}}
{\mbox{\boldmath$\scriptscriptstyle#1$}}\else
\hbox{\boldmath$\textstyle#1$}\fi}

\newcommand{\eq}[1]{(\ref{#1})}

\newcommand{\cbar}{\overline{c}}
\newcommand{\thetabar}{\overline{\theta}}
\newcommand{\thetatilde}{\widetilde{\theta}}

\newcommand{\vk}{\vector{k}}
\newcommand{\vu}{\vector{u}}
\newcommand{\vv}{\vector{v}}
\newcommand{\vx}{\vector{x}}

\newcommand{\ubar}{\overline{u}}
\newcommand{\vbar}{\overline{v}}

\newcommand{\advection}[1]{\mathcal{A}\left[#1\right]}
\newcommand{\advectionNum}[1]{\widetilde{\mathcal{A}}\left[#1\right]}
\newcommand{\rhs}[1]{\mathcal{R}\left[#1\right]}
\newcommand{\source}[1]{S\left(#1\right)}
\newcommand{\Id}{{\rm Id}}

\newcommand{\pprime}{p'}
\newcommand{\half}{\frac{1}{2}}
\newcommand{\quart}{\frac{1}{4}}
\newcommand{\eightth}{\frac{1}{8}}

\newcommand{\ie}{\emph{i.e.}}

\newcommand{\Nsq}{N^2}
\newcommand{\Nscsq}{(\tau N)^2}

\newcommand{\dt}{\Delta t}
\newcommand{\dz}{\Delta z}
\newcommand{\chibar}{\overline{\chi}}
\newcommand{\chitilde}{{\widetilde \chi}}

\newcommand{\pp}[2]{\frac{\partial #1}{\partial #2}}
\newcommand{\ppn}[3]{\frac{\partial^{#1} #2}{\partial #3^{#1}}}

\newcommand{\bigoh}[1]{\mathcal{O}\left(#1\right)}
\newcommand{\littleoh}[1]{{\scriptstyle\mathcal{O}}\left(#1\right)}

\newcommand{\reals}{\mathbb{R}}

\newcommand{\eps}{\varepsilon}

\newcommand{\Lopfirst}{{\cal L}^{\rm 1st}}

\newcommand{\pibar}{\overline{\pi}}

% ===========================================================================
% ===========================================================================
% ===========================================================================
% ===========================================================================

\title{Semi-implicit gravity}
\author{T.~Benacchio, MetOffice, Exeter, UK\\ 
        R.~Klein, Mathematics \& Informatics, Freie Universit\"at Berlin}

\begin{document}

\maketitle

Hi Tommaso, \\

in dreaming up the semi-implicit gravity version in the appendix of your
thesis, I think we missed one crucial point. Reconsidering, I came to the
conclusion that it is the ``explicit part of the implicit step''�in the
trapezoidal rule that controls vertical velocities, and thus stability, 
for large time steps. I'll try to highlight this in the following short 
note. \\

Best, \\

Rupert

% ===========================================================================
% ===========================================================================
% ===========================================================================

\section{The second step of the implicit trapezoidal rule}
\label{sec:trapezoidal}

Focusing just on the crucial linear terms in the governing equations that
are responsible for internal waves, we have -- for advection of the background 
potential temperature and for the vertical momentum, 
%
\begin{eqnarray}
w_t 
  & = 
    & - \frac{\theta}{\Gamma}\left(\pi_z + \Gamma g \chi\right)
      \\
\chi_t
  & =
    & - w \frac{d\overline{\chi}}{dz}
\end{eqnarray}
%
here $\pi$ is Exner pressure, $\Gamma = (\gamma - 1) / \gamma$ with $\gamma$
the isentropic exponent, $\chi = 1/\theta$, and the prefactor $\theta$ in the 
first equation is frozen in for the linearization. Suppose that an explicit 
Euler forward step for these equations over $\Delta t / 2$ as well as 
advection of all quantities have been taken care of in operator splitting-like 
steps, so that all that is left to do is an implicit, backward Euler step over 
$\tau = \Delta t / 2$. 

Suppose that $(\chi^*, w^*, \pi^*)$ denote the state of the relevant solution 
variables at the start of the implicit substep and 
$(\delta\chi, \delta w, \delta\pi)$ the respective updates over the implicit
step. Then,
%
\begin{eqnarray}
\label{eq:deltaw1}
w^{n+1} - w^* \equiv \delta w 
  & = 
    & - \tau \, \frac{\theta}{\Gamma}\left([\pi^*+\delta\pi]_z + \Gamma g [\chi^* + \delta\chi]\right)
      \\
\label{eq:deltachi1}
\chi^{n+1} - \chi^* \equiv \delta\chi
  & =
    & - \tau\, (w^* + \delta w) \frac{d\overline{\chi}}{dz}\,.
\end{eqnarray}
%

Now we reorder explicit and implicit contributions, replace $\delta\chi$ in 
\eq{eq:deltaw1} using \eq{eq:deltachi1} and solve for $\delta w$, 
%
\begin{eqnarray}
w^{n+1} - w^* = \delta w 
  & = 
    & - \tau \, \frac{\theta}{\Gamma}\left(\pi^*_z + \Gamma g \chi^*\right)
      - \tau \, \frac{\theta}{\Gamma}\left(\delta\pi_z + \Gamma g \delta\chi\right)
      \\
  & =
    & - \tau \, \frac{\theta}{\Gamma}\left(\pi^*_z + \Gamma g \chi^*\right)
      - \tau \, \frac{\theta}{\Gamma}\left(\delta\pi_z 
      - \tau \, \Gamma g (w^* + \delta w) \frac{d\overline{\chi}}{dz}\right)
      \\
\delta w \left(1 - \tau^2 g \theta \frac{d\overline{\chi}}{dz}\right)
  & =
    & - \tau \, \frac{\theta}{\Gamma}\left(\pi^*_z + \Gamma g \chi^*\right)
      + w^* \tau^2 g \theta \frac{d\overline{\chi}}{dz}
      - \tau \, \frac{\theta}{\Gamma} \delta\pi_z \,.
\end{eqnarray}
%
Letting
%
\begin{equation}
\Nsq = - g \theta \frac{d\overline{\chi}}{dz}
\end{equation}
%
denote the square of the buoyancy-frequency, we have
%
\begin{equation}
\label{eq:deltaw2}
\delta w = \frac{1}{1 + \Nscsq} 
\left(- \tau \frac{\theta}{\Gamma}\left(\pi^*_z + \Gamma g \chi^*\right)
      - w^* \Nscsq\right)
      - \tau \frac{\theta/\Gamma}{1 + \Nscsq} \delta\pi_z
\end{equation}
%
and the update formular for $\chi$ subsequently follows from \eq{eq:deltachi1}.

Consider now the first term on the right in \eq{eq:deltaw2}, which contains all
explicit contributions to the velocity update, and its scaling for large time
steps $\tau \to \infty$. The first term in the bracket scales linearly with
$\tau$, so that it vanishes in the limit. The second term in the bracket involves
$(\tau N)^2$, however, and therefore the limit reads
%
\begin{equation}
\delta w\big|_{\tau \to \infty} = -w^*\, .
\end{equation}
%
That is, the first thing the explicit part of the implicit step does is to 
set the vertical velocity to zero in that limit! This is clearly a stabilizing
effect that can capture any large excursion in $w^*$ that may have incurred 
during the explicit half-time step of the trapezoidal rule. 

Tommaso, I think we left this bit out in our earlier attempts. I have implemented
a variant of this in my scheme and see the stabilization of vertical velocity 
very nicely for $N \Delta t$ as large as five. The large scale horizontal
domain with a hydrostatic gravity wave still does not quite work yet, but I have
two more things to try: (i) locally hydrostatic initialization of pressure - which
in my current setup I do not have, and (ii) a preconditioner that addresses the vertical
part of the Poisson problem, instead of preconditioning only through division by
the diagonal element of the Poisson-matrix. Will keep you posted. 

Best, 

Rupert

\newpage

% ===========================================================================
% ===========================================================================
% ===========================================================================

\section{Implicit trapezoidal rule -- variant 2}
\label{sec:trapezoidal2}

% ===========================================================================
% ===========================================================================

\subsection{Linear oscillator}

To understand the mechanism of stabilization for arbitrary time steps in the
implicit trapezoidal rule, we consider 
%
\begin{equation}
\ddot y + y = 0
\qquad\text{or rather}\qquad
\begin{array}{rcr}
\dot y 
  & = 
    & v
      \\
\dot v 
  & =
    & - y
\end{array}\,.
\end{equation}
%
The implicit trapezoidal rule for the system version of this problem over
one time step reads
%
\begin{equation}\label{eq:ImplicitTrapezoidalOscillator}
\begin{array}{rcr}
\dss y^{n+1} - y^n 
  & = 
    & \dss \frac{\dt}{2} \left(v^{n+1} +  v^n\right)
      \\[10pt]
\dss v^{n+1} - v^n 
  & =
    & \dss - \frac{\dt}{2} \left(y^{n+1} +  y^n\right)
\end{array}\,.
\end{equation}
%
We solve for $v^{n+1}$ first:
%
\begin{equation}\label{eq:ImplicitTrapezoidalStepOscillator}
\begin{array}{rcl}
\dss v^{n+1} 
  & = 
    & \dss v^n - \frac{\dt}{2} \left(y^n + \frac{\dt}{2}\left(v^{n+1} + v^n\right) + y^n\right) 
      \\[10pt]
\dss v^{n+1} \left(1 + \frac{(\dt)^2}{4}\right)
  & =
    & \dss v^n \left(1 - \frac{(\dt)^2}{4}\right) - \dt\, y^n
      \\
\dss v^{n+1} 
  & =
    & \dss \frac{1 - \frac{(\dt)^2}{4}}{1 + \frac{(\dt)^2}{4}} \  v^n 
          - \frac{\dt}{1 + \frac{(\dt)^2}{4}}\, y^n
\end{array}
\end{equation}
or
%
\begin{equation}
v^{n+1} - v^n  \equiv \delta v
= - \ 2 v^n  \ \frac{\frac{(\dt)^2}{4}}{1 + \frac{(\dt)^2}{4}} 
          - \frac{\dt}{1 + \frac{(\dt)^2}{4}}\, y^n
\end{equation}
%
%
Similarly, 
%
\begin{equation}
\begin{array}{rcl}
\dss y^{n+1} 
  & = 
    & \dss y^n + \frac{\dt}{2} \left(v^n - \frac{\dt}{2}\left(y^{n+1} + y^n\right) + v^n\right) 
      \\[10pt]
\dss y^{n+1} \left(1 + \frac{(\dt)^2}{4}\right)
  & =
    & \dss y^n \left(1 - \frac{(\dt)^2}{4}\right) + \dt\, v^n
      \\
\dss y^{n+1} 
  & =
    & \dss \frac{1 - \frac{(\dt)^2}{4}}{1 + \frac{(\dt)^2}{4}} \  y^n 
          + \frac{\dt}{1 + \frac{(\dt)^2}{4}}\, v^n
\end{array}
\end{equation}
%


Clearly, 
%
\begin{equation}
\lim\limits_{\dt\to\infty} (y^{n+1}, n^{n+1}) = - (y^n, v^n)\, ,
\end{equation}
%
and this characterizes the energy-preserving, oscillatory nature, and
unconditional neutrality of the scheme in the large-time step limit. 

Can we understand the implicit trapezoidal rule as a two-step scheme 
involving a forward Euler predictor and a backward Euler corrector? 
Let's see.

Forward Euler predictor:
%
\begin{equation}
\begin{array}{rcr}
\dss y^* - y^n 
  & = 
    & \dss \frac{\dt}{2} \, v^n
      \\[10pt]
\dss v^* - v^n 
  & =
    & \dss - \frac{\dt}{2} \, y^n
\end{array}\,.
\end{equation}
%

Backward Euler corrector:
%
\begin{equation}
\begin{array}{rcr}
\dss y^{n+1} - y^*
  & = 
    & \dss \frac{\dt}{2} \, v^{n+1}
      \\[10pt]
\dss v^{n+1} - v^* 
  & =
    & \dss - \frac{\dt}{2} \, y^{n+1}
\end{array}\,.
\end{equation}
%
Obviously, adding the two equation sets we get back the implicit 
trapezoidal rule from \eq{eq:ImplicitTrapezoidalOscillator}. Solving for
the new time data, however, given the predicted ones, we obtain
%
\begin{equation}
\begin{array}{rcl}
\dss v^{n+1} 
  & = 
    & \dss v^* - \frac{\dt}{2} \left(y^* + \frac{\dt}{2}\, v^{n+1}\right) 
      \\[10pt]
\dss v^{n+1} \left(1 + \frac{(\dt)^2}{4}\right)
  & =
    & \dss v^* - \frac{\dt}{2}\, y^*
      \\
\dss v^{n+1} 
  & =
    & \dss \frac{1}{1 + \frac{(\dt)^2}{4}} \  v^* 
          - \frac{1}{2}\frac{\dt}{1 + \frac{(\dt)^2}{4}}\, y^*
\end{array}
\end{equation}
%
For the increment within the half timestep we have
%
\begin{equation}
v^{n+1} - v^* \equiv \delta v 
= - \ v^* \ \frac{\frac{(\dt)^2}{4}}{1 + \frac{(\dt)^2}{4}}  
          - \frac{1}{2}\frac{\dt}{1 + \frac{(\dt)^2}{4}}\, y^*
\end{equation}
%



To get back to the original formula from 
\eq{eq:ImplicitTrapezoidalStepOscillator} we re-insert the predictor 
step to obtain
%
\begin{equation}
\begin{array}{rcl}
\dss v^{n+1} 
  & =
    & \dss \frac{1}{1 + \frac{(\dt)^2}{4}} \  v^* 
          - \frac{1}{2}\frac{\dt}{1 + \frac{(\dt)^2}{4}}\, y^*
      \\[15pt]
  & =
    & \dss \frac{1}{1 + \frac{(\dt)^2}{4}} \left(v^n - \frac{\dt}{2} y^n\right) 
          - \frac{1}{2}\frac{\dt}{1 + \frac{(\dt)^2}{4}} \left(y^n + \frac{\dt}{2} v^n\right)
      \\[15pt]
  & =
    & \dss \frac{1 - \frac{(\dt)^2}{4}}{1 + \frac{(\dt)^2}{4}} \,  v^n
          - \frac{\dt}{1 + \frac{(\dt)^2}{4}}\, y^n
      \\
\end{array}
\end{equation}
%
And analogously for $y^{n+1}$, i.e., 
%
\begin{equation}
y^{n+1} = \frac{1 - \frac{(\dt)^2}{4}}{1 + \frac{(\dt)^2}{4}} \,  y^n
          + \frac{\dt}{1 + \frac{(\dt)^2}{4}}\, v^n
\end{equation}
%

We distinguish the cases of $(y^n, v^n) = \bigoh{1}$ and $y^n = 0$ or
$v^n = 0$ and consider the limit $\dt\to\infty$: 
%
\begin{enumerate}

\item Obviously, when $(y^n, v^n) = \bigoh{1}$, we have
%
\begin{equation}
\left.y^{n+1}\right|_{\dt\to\infty} = - y^n\,.
\qquad\text{and}\qquad
\left.v^{n+1}\right|_{\dt\to\infty} = - v^n
\end{equation}
%

\item However, if the respective initial datum is zero, the limiting behavior
changes as follows. Suppose $y^n = 0$, but $v^n \not= 0$. Then, 
%
\begin{equation}
y^{n+1} = \frac{\dt}{1 + \frac{(\dt)^2}{4}}\, v^n
\qquad\text{and}\qquad
v^{n+1} = \frac{1 - \frac{(\dt)^2}{4}}{1 + \frac{(\dt)^2}{4}} \,  v^n \to - v^n\,.
\end{equation}
%

\end{enumerate}
%


Now, can we also understand the implicit trapezoidal rule as a two-step scheme 
involving a first the backward and second the forward Euler step? 
Let's see.

Backward Euler step:
%
\begin{equation}
\begin{array}{rcr}
\dss y^{*} - y^n
  & = 
    & \dss \frac{\dt}{2} \, v^{*}
      \\[10pt]
\dss v^{*} - v^n 
  & =
    & \dss - \frac{\dt}{2} \, y^{*}
\end{array}\,.
\end{equation}
%

Forward Euler step:
%
\begin{equation}\label{eq:FowardEulerII}
\begin{array}{rcr}
\dss y^{n+1} - y^* 
  & = 
    & \dss \frac{\dt}{2} \, v^*
      \\[10pt]
\dss v^{n+1} - v^* 
  & =
    & \dss - \frac{\dt}{2} \, y^*
\end{array}\,.
\end{equation}
%
Solving for
the intermediate time data first we obtain
%
\begin{equation}
\begin{array}{rcl}
\dss v^{*} 
  & = 
    & \dss v^n - \frac{\dt}{2} y^* 
      \\
\dss y^{*}
  & = 
    & \dss y^n + \frac{\dt}{2} v^* = y^n + \frac{\dt}{2} \left(v^n - \frac{\dt}{2} y^*\right)
      \\[10pt]
\dss y^{*} \left(1 + \frac{(\dt)^2}{4}\right)
  & =
    & \dss y^n + \frac{\dt}{2}\, v^n
      \\
\dss y^{*} 
  & =
    & \dss \frac{1}{1 + \frac{(\dt)^2}{4}} \  
          \left(y^n + \frac{\dt}{2}\, v^n\right)
\end{array}
\end{equation}
%
and then, 
%
\begin{equation}
\begin{array}{rcl}
\dss v^{*} 
  & = 
    & \dss v^n - \frac{\dt}{2} y^* 
      \\
  & = 
    & \dss v^n - \frac{\frac{\dt}{2}}{1 + \frac{(\dt)^2}{4}} 
          \left(y^n + \frac{\dt}{2} v^n\right)
      \\[10pt]
  & =
    & \dss \frac{1}{1 + \frac{(\dt)^2}{4}} \left(v^n - \frac{\dt}{2} y^n\right)
\end{array}
\end{equation}
%
Next we insert into the forward Euler step in \eq{eq:FowardEulerII} to obtain,
%
\begin{equation}
\begin{array}{rcl}
\dss v^{n+1} 
  & = 
    & \dss v^* - \frac{\dt}{2} y^* 
      \\
  & = 
    & \dss \frac{1}{1 + \frac{(\dt)^2}{4}} 
          \left(v^n - \frac{\dt}{2} y^n  - \frac{\dt}{2}\left(y^n + \frac{\dt}{2}v^n\right)
          \right)
      \\[10pt]
  & =
    & \dss \frac{1 - \frac{(\dt)^2}{4}}{1 + \frac{(\dt)^2}{4}} v^n 
          - \frac{\dt}{1 + \frac{(\dt)^2}{4}} y^n
\end{array}
\end{equation}
%
and that is the same as found in \eq{eq:ImplicitTrapezoidalStepOscillator}.

% ===========================================================================

\subsubsection{Summary}

Euler forward over $\Delta t$:
%
\begin{equation}\label{eq:FowardEulerIII}
\begin{array}{rcr}
\dss y^{\rm new}  
  & = 
    & \dss y^{\rm old} + \dt \, v^{\rm old}
      \\[10pt]
\dss v^{\rm new}  
  & =
    & \dss v^{\rm old} - \dt \, y^{\rm old}
\end{array}\,.
\end{equation}
%

Euler backward over $\Delta t$:
%
\begin{equation}
\begin{array}{rcl}
\dss y^{\rm new} 
  & =
    & \dss \frac{1}{1 + (\dt)^2} \  
          \left(y^{\rm old} + \dt\, v^{\rm old}\right)
       \ = \ y^{\rm old} - \frac{(\dt)^2}{1 + (\dt)^2} y^{\rm old} + \frac{\dt}{1 + (\dt)^2} v^{\rm old}
      \\[15pt]
\dss v^{\rm new} 
  & =
    & \dss \frac{1}{1 + (\dt)^2} \  
          \left(v^{\rm old} - \dt\, y^{\rm old}\right)
       \ = \ v^{\rm old} - \frac{(\dt)^2}{1 + (\dt)^2} v^{\rm old} - \frac{\dt}{1 + (\dt)^2} y^{\rm old}
\end{array}
\end{equation}
%

Implicit trapezoidal rule over $\dt$:
%
\begin{equation}
\begin{array}{rcl}
\dss y^{\rm new} 
  & =
    & \dss \frac{1 - \frac{(\dt)^2}{4}}{1 + \frac{(\dt)^2}{4}} y^{\rm old} 
          + \frac{\dt}{1 + \frac{(\dt)^2}{4}} v^{\rm old}
       \ = \ y^{\rm old} - \frac{2 \frac{(\dt)^2}{4}}{1 + \frac{(\dt)^2}{4}} y^{\rm old} 
         + \frac{\dt}{1 + \frac{(\dt)^2}{4}} v^{\rm old}
      \\[15pt]
\dss v^{\rm new} 
  & =
    & \dss \frac{1 - \frac{(\dt)^2}{4}}{1 + \frac{(\dt)^2}{4}} v^{\rm old} 
          - \frac{\dt}{1 + \frac{(\dt)^2}{4}} y^{\rm old}
       \ = \ v^{\rm old} - \frac{2 \frac{(\dt)^2}{4}}{1 + \frac{(\dt)^2}{4}} v^{\rm old} 
         - \frac{\dt}{1 + \frac{(\dt)^2}{4}} y^{\rm old}
\end{array}
\end{equation}
%

% ===========================================================================
% ===========================================================================

\subsection{Implicit gravity}

% ===========================================================================

\subsubsection{Explicit part of the implicit trapezoidal rule}

Focusing just on the crucial linear terms in the governing equations that
are responsible for internal waves, we have -- for advection of the background 
potential temperature and for the vertical momentum, 
%
\begin{eqnarray}
w_t 
  & = 
    & - \frac{\theta}{\Gamma}\left(\pi_z + \Gamma g \chi\right)
      \\
\chi_t
  & =
    & - w \frac{d\overline{\chi}}{dz}
\end{eqnarray}
%
here $\pi$ is Exner pressure, $\Gamma = (\gamma - 1) / \gamma$ with $\gamma$
the isentropic exponent, $\chi = 1/\theta$, and the prefactor $\theta$ in the 
first equation is frozen in for the linearization. Suppose we wish to integrate
these equations using the implicit trapezoidal rule over a time step $\dt$. 
Then,
%
\begin{eqnarray}
\label{eq:deltaw1a}
w^{n+1} - w^n \equiv \delta w 
  & = 
    & - \dt \, \frac{\theta}{\Gamma}
     \left(\left[\pi^n+\frac{\delta\pi}{2}\right]_z + \Gamma g 
           \left[\chi^n + \frac{\delta\chi}{2}\right]\right)
      \\
\label{eq:deltachi1a}
\chi^{n+1} - \chi^n \equiv \delta\chi
  & =
    & - \dt\, \left(w^n + \frac{\delta w}{2}\right) \frac{d\overline{\chi}}{dz}\,.
\end{eqnarray}
%

Now we reorder explicit and implicit contributions, replace $\delta\chi$ in 
\eq{eq:deltaw1a} using \eq{eq:deltachi1a} and solve for $\delta w$, 
%
\begin{eqnarray}
w^{n+1} - w^n = \delta w 
  & = 
    & - \dt \, \frac{\theta}{\Gamma}\left(\pi^n_z + \Gamma g \chi^n\right)
      - \frac{\dt}{2} \, \frac{\theta}{\Gamma}\left(\delta\pi_z + \Gamma g \delta\chi\right)
      \\
  & =
    & - \dt \, \frac{\theta}{\Gamma}\left(\pi^n_z + \Gamma g \chi^n\right)
      - \frac{\dt}{2} \, \frac{\theta}{\Gamma}\left(\delta\pi_z 
      - \dt \, \Gamma g \left(w^n + \frac{\delta w}{2}\right) \frac{d\overline{\chi}}{dz}\right)
      \\
\delta w \left(1 - \frac{(\dt)^2}{4} g \theta \frac{d\overline{\chi}}{dz}\right)
  & =
    & - \dt \, \frac{\theta}{\Gamma}\left(\pi^n_z + \Gamma g \chi^n\right)
      + w^n \frac{(\dt)^2}{2} g \theta \frac{d\overline{\chi}}{dz}
      - \frac{\dt}{2} \, \frac{\theta}{\Gamma} \delta\pi_z \,.
\end{eqnarray}
%
Letting
%
\begin{equation}
\Nsq = - g \theta \frac{d\overline{\chi}}{dz}
\end{equation}
%
denote the square of the buoyancy-frequency, we have
%
\begin{equation}
\label{eq:deltaw2b}
\delta w = \frac{1}{1 + \left(\frac{N\dt}{2}\right)^2} 
\left(- \dt \frac{\theta}{\Gamma}\left(\pi^n_z + \Gamma g \chi^n\right)
      - 2\, w^n \left(\frac{N\dt}{2}\right)^2\right)
      - \frac{\dt}{2} \frac{\theta/\Gamma}{1 + \left(\frac{N\dt}{2}\right)^2} \delta\pi_z
\end{equation}
%
and the update formula for $\chi$ subsequently follows from \eq{eq:deltachi1a}. We summarize
these results in the following formulae for the new time level data, 
%
\begin{eqnarray}
\label{eq:deltaw3}
w^{n+1} 
  & = 
    & \frac{1 - \left(\frac{N\dt}{2}\right)^2}{1 + \left(\frac{N\dt}{2}\right)^2}\, w^n
-\frac{\dt}{1 + \left(\frac{N\dt}{2}\right)^2} 
\left( \frac{\theta}{\Gamma}\left(\pi^n_z + \Gamma g \chi^n\right)\right)
      - \frac{\dt}{2} \frac{\theta/\Gamma}{1 + \left(\frac{N\dt}{2}\right)^2} \delta\pi_z
\\
\label{eq:deltaChi3}
\chi^{n+1} 
  & =
    & \frac{1 - \left(\frac{N\dt}{2}\right)^2}{1 + \left(\frac{N\dt}{2}\right)^2}\, \chi^n
-\frac{\dt}{1 + \left(\frac{N\dt}{2}\right)^2} \frac{d\overline{\chi}}{dz} \, w^n\,.
\end{eqnarray}
%
It is also interesting to note the change in time of the momentum source term 
for frozen $\pi^n$ in the following fashion, 
%
\begin{equation}
\left(\chi + \frac{\pi_z}{\Gamma g}\right)^{n+1}
=
\frac{1 - \left(\frac{N\dt}{2}\right)^2}{1 + \left(\frac{N\dt}{2}\right)^2}\, 
\left(\chi + \frac{\pi_z}{\Gamma g}\right)^n
-\frac{\dt}{1 + \left(\frac{N\dt}{2}\right)^2} \frac{d\overline{\chi}}{dz} \, w^n\,.
\end{equation}
%


Consider now the first term on the right in \eq{eq:deltaw3}, which contains all
explicit contributions to the velocity update, and its scaling for large time
steps $\dt \to \infty$. The first term in the bracket scales linearly with
$\dt$, so that it vanishes in the limit. The second term in the bracket involves
$(\dt N)^2$, however, and therefore the limit reads
%
\begin{equation}
\delta^{\rm expl} w\big|_{\dt \to \infty} = - 2 w^n
\qquad\text{or}\qquad
w^{n+1} = - w^n\, .
\end{equation}
%
That is, the first thing the explicit part of the implicit step does is to 
impose a timestep--to--timestep oscillation as expected. This is the energy-preserving
property of the implicit trapezoidal rule in the linear case.

% ===========================================================================

\subsubsection{Application to advective flux calculations}
\label{sssec:AdvectiveFluxes}

Here we have to consider the joint evolution of $P$ and $P\chi = \rho$ in the
vertical split half time step, i.e.,
%
\begin{eqnarray}
\label{eq:LinearizedConservativeChiEqn}
(P\chi)_t
  & = 
    & - (P\chibar w)_z
      \\
(P)_t
  & = 
    & - (P w)_z
\end{eqnarray}
%
and utilize the implicit trapezoidal rule as worked out in \eq{eq:deltaw3} 
to determine the advecting fluxes $(Pw)$. Thus, 
%
\begin{eqnarray}
\label{eq:ConservativeChiUpdate1}
(P\chi)_j^{n+\half} - (P\chi)_j^n
  & = 
    & - \frac{\dt}{2\dz} \left( (P\chibar w)_{j+\half}^{n+\quart} - (P\chibar w)_{j-\half}^{n+\quart} \right)
      \\
P_j^{n+\half} - P_j^n
  & = 
    & - \frac{\dt}{2\dz} \left( (P w)_{j+\half}^{n+\quart} - (P w)_{j-\half}^{n+\quart} \right)
\end{eqnarray}
%
What I want to show is that the update $\chi_j^{n+\half}-\chi_j^n$ is as 
nicely controlled for large $\dt$ as is the nonconservative update from
\eq{eq:deltaChi3}.

Rewrite \eq{eq:ConservativeChiUpdate1} as
%
\begin{eqnarray}
P^{n+\quart*}_j \left(\chi_j^{n+\half} - \chi_j^{n}\right)
  & =  
    & - \left(P^{n+\half}_j - P^n_j\right) \chi_j^{n+\quart*}
      - \frac{\dt}{2\dz} 
        \left( (P\chibar w)_{j+\half}^{n+\quart} - (P\chibar w)_{j-\half}^{n+\quart} 
        \right)
      \\[5pt]
  & =  
    & - \frac{\dt}{2\dz} 
        \left( (P w)_{j+\half}^{n+\quart} (\chibar_{j+\half} - \chi_j^{n+\quart*}) 
             - (P w)_{j-\half}^{n+\quart} (\chibar_{j-\half} - \chi_j^{n+\quart*})
        \right)
      \\[5pt]
  & =  
    & - \frac{\dt}{2\dz} \  
        \frac{1}{2}\left((P w)_{j+\half}^{n+\quart} + (P w)_{j-\half}^{n+\quart}\right) 
        \left(\chibar_{j+\half} - \chibar_{j-\half}\right)
      \\[5pt]
  &   
    & - \frac{\dt}{2\dz} 
        \left( \frac{1}{2}\left(\chibar_{j+\half} - 2\chi_j^{n+\quart*} + \chibar_{j-\half}\right)
               \left( (P w)_{j+\half}^{n+\quart} - (P w)_{j-\half}^{n+\quart} \right)
        \right)
      \nonumber\\[5pt]
  & =  
    & - \frac{\dt}{2\dz} \  
        \frac{1}{2}\left((P w)_{j+\half}^{n+\quart} + (P w)_{j-\half}^{n+\quart}\right) 
        \left(\chibar_{j+\half} - \chibar_{j-\half}\right)
      \\[5pt]
  &   
    & + \frac{\dt}{2} (\dz)^2 \frac{2}{\dt}\left( P_j^{n+\half} - P_j^{n} \right) 
        \frac{1}{2(\dz)^2}\left(\chibar_{j+\half} - 2\chi_j^{n+\quart*} + \chibar_{j-\half}\right)
      \nonumber\\[5pt]
 \label{eq:ConservativeChiUpdate6}
  & =  
    & - \frac{\dt}{2\dz} \  
        \frac{1}{2}\left((P w)_{j+\half}^{n+\quart} + (P w)_{j-\half}^{n+\quart}\right) 
        \left(\chibar_{j+\half} - \chibar_{j-\half}\right)
      \\[5pt]
  &   
    & - \left( P_j^{n+\half} - P_j^{n} \right) \left(\chi_j^{n+\quart*} - \chibar_j^n\right)
      \nonumber\\[5pt]
  &   
    & + \frac{\dt}{16} (\dz)^2 \frac{2}{\dt}\left( P_j^{n+\half} - P_j^{n} \right) 
        \frac{4}{(\dz)^2}\left(\chibar_{j+\half} - 2\chibar_j^n + \chibar_{j-\half}\right)
      \nonumber\\[5pt]
 \label{eq:ConservativeChiUpdate7}
  & = 
    & - \frac{\dt}{2\dz} \  
        \frac{1}{2}\left((P w)_{j+\half}^{n+\quart} + (P w)_{j-\half}^{n+\quart}\right) 
        \left(\chibar_{j+\half} - \chibar_{j-\half}\right)
      \\[5pt]
  &   
    & - \left( P_j^{n+\half} - P_j^{n} \right) \left(\chi_j^{n+\quart*} - \chibar_j^n\right)
      + \frac{\dt(\dz)^2}{16} \left(\pp{P}{t}\right)_{j}^{n+\quart} 
        \left(\ppn{2}{\chibar}{z}\right)_{j}^{n}
      \nonumber
\end{eqnarray}
%
where we have used the abbreviations
%
\begin{equation}
X^{n+\quart*} = \frac{1}{2}\left(X^{n+\half} + X^{n}\right)\,.
\end{equation}
%
While the last term in \eq{eq:ConservativeChiUpdate7} 
is arguably of higher order and may be neglected unless
we want to demonstrate some rigorous bound on $\chi$ or the like, the next
to last term is not negligible. It is the very kind of contribution that 
induces us to carry the $P$-variable along in our scheme even if we solve
the pseudo-incompressible equations despite the fact that in that equation
set, $\partial_t P \equiv 0$: In each split step of our OpSplit procedure, 
that derivative will be nonzero and of order unity in general. 

We understand that our scheme is better than one may think at this point 
if instead of considering only the linearized update for $P\chi$ according
to \eq{eq:LinearizedConservativeChiEqn}, we consider the full nonlinear 
expression
%
\begin{equation}
\pp{P\chi}{t} = - \pp{(P\chi w)}{z}
\end{equation}
%
which turns \eq{eq:ConservativeChiUpdate6} into 
%
\begin{eqnarray}
\frac{2}{\dt} \left(\chi_j^{n+\half} - \chi_j^{n}\right)
  & =  
    & - \frac{1}{2P^{n+\quart*}_j}\left((P w)_{j+\half}^{n+\quart} + (P w)_{j-\half}^{n+\quart}\right) 
        \frac{\chi^{n+\quart}_{j+\half} - \chi^{n+\quart}_{j-\half}}{\dz}      
      \\[5pt]
  &   
    & + \frac{(\dz)^2}{8}  \frac{2}{\dt}\left( P_j^{n+\half} - P_j^{n} \right) 
        \frac{4}{(\dz)^2}\left(\chi^{n+\quart}_{j+\half} - 2\chi^{n+\quart*}_j + \chi^{n+\quart}_{j-\half}\right)
      \nonumber\\
\label{eq:ConservativeChiUpdate8}
  & =  
    & - \frac{1}{2P_j^{n+\quart*}}\left((P w)_{j+\half}^{n+\quart} + (P w)_{j-\half}^{n+\quart}\right) 
        \frac{\chi^{n+\quart}_{j+\half} - \chi^{n+\quart}_{j-\half}}{\dz}      
      \\[5pt]
  &   
    & + \frac{(\dz)^2}{8} \left(\pp{P}{t}\right)_{j}^{n+\quart} 
        \left(\ppn{2}{\chi}{z}\right)_{j}^{n+\quart} + \littleoh{(\dz)^2}\,.
      \nonumber
\end{eqnarray}
%
For a second-order MUSCL-scheme we can do even better. In that case, reconstruction
at $t^{n+\quart}$ is piecewise linear, $\chi^{n+\quart*}_j = \chi^{n+\quart}_j + \bigoh{(\dt)^2}$, 
and the edge values $\chi^{n+\quart}_{j+\half}$ are determined on an upwind basis. 

Suppose, for case 1, that $\chi$ is smooth and does not have an extremum, and that 
$(P w)_{j\pm\half}^{n+\quart} > 0$, then
%
\begin{eqnarray}
\chi^{n+\quart}_{j+\half} - 2\chi^{n+\quart*}_j + \chi^{n+\quart}_{j-\half}
  & =
    & \chi^{n+\quart}_{j,+} - 2\chi^{n+\quart*}_j + \chi^{n+\quart}_{j-1,+}
      \\
  & =
    & \chi^{n+\quart}_{j,+} - 2\chi^{n+\quart}_j + \chi^{n+\quart}_{j,-}
      \\
  & 
    & + \left(\chi^{n+\quart}_{j-1,+} - \chi^{n+\quart}_{j,-}\right)
      - 2 \left(\chi^{n+\quart*}_{j} - \chi^{n+\quart}_{j}\right)
      \\
  & =
    & \left(\chi^{n+\quart}_{j-1,+} - \chi^{n+\quart}_{j,-}\right) + \bigoh{(\dt)^2}
      \\
  & =
    & \left(\chi^{n+\quart}_{j-1} + \frac{\dz}{2} S_{j-1} 
          - \chi^{n+\quart}_{j} + \frac{\dz}{2} S_{j} \right) + \bigoh{(\dt)^2}
      \\
  & =
    & \dz \left(\frac{1}{2} \left(S_{j-1} +  S_{j}\right) 
               - \frac{\chi^{n+\quart}_{j} - \chi^{n+\quart}_{j-1}}{\dz}
          \right) + \bigoh{(\dt)^2}
      \\
  & = 
    & \bigoh{(\dt)^2 + (\dz)^2}
\end{eqnarray}
%

Issues to be addressed: This latter result is not ``better'', as we have
already assumed in the previous estimate that this expression is basically
$(\dz)^2(\partial^2 \chi / \partial z^2)$. What we do have to worry about
are extrema of $\chi$, where the slope reconstruction deteriorates and we
only have a spacial first-order accurate reconstruction, and about the
$\bigoh{(\dt)^2}$ term. \\
That term can deteriorate for large $\dt$, which is the current focus! 
Checking where it comes from, we have to make sure that both 
$\chi_{j,\pm}^{n+\quart}$ (from the reconstruction) and 
$\chi_{j,\pm}^{n+\quart*}$ (from the time update within cell $j$) are
controlled as $\dt \to \infty$ as displayed above in working out the 
explicit part of the semi-implicit time integrator. 

Let us, for the moment, not worry about the second derivative term that
is an unwanted left-over imprint from the time changes of $P$, but rather
focus on the dominant part of the $\chi$-update in 
\eq{eq:ConservativeChiUpdate8}.  
%
\begin{eqnarray}
\frac{2}{\dt} \left(\chi_j^{n+\half} - \chi_j^{n}\right)
  & =  
    & - \frac{1}{2P_j^{n+\quart*}}\left((P w)_{j+\half}^{n+\quart} + (P w)_{j-\half}^{n+\quart}\right) 
        \frac{\chi^{n+\quart}_{j+\half} - \chi^{n+\quart}_{j-\half}}{\dz}      
      + \littleoh{(\dz)^2}\,.
\end{eqnarray}
%
Following the MUSCL strategy translated into the implicit trapezoidal 
context, we insert a discretization of
%
\begin{eqnarray}
w_t 
  & = 
    & - \frac{\theta}{\Gamma}\left(\pi_z^n + \Gamma g \chi\right)
      \\
\chi_t 
  & = 
    & - w \frac{d\chibar}{dz}
\end{eqnarray}
%
over a half time step and then utilize $w^{n+\quart} = \half \left(w^{n+\quart} + w^n\right)$
to extract the half of a half time step level vertical velocities. According to the 
previous considerations in \eq{eq:deltaw3}, this yields
%
\begin{eqnarray}
\label{eq:deltaw4}
w^{n+\half} 
  & = 
    & \frac{1 - \left(\frac{N\dt}{4}\right)^2}{1 + \left(\frac{N\dt}{4}\right)^2}\, w^n
-\frac{\dt/2}{1 + \left(\frac{N\dt}{4}\right)^2} 
\left( \frac{\theta}{\Gamma}\left(\pi^n_z + \Gamma g \chi^n\right)\right)
\\
w^{n+\quart} = \frac{1}{2}\left(w^{n+\half} + w^{n}\right)
  & = 
    & \frac{1}{1 + \left(\frac{N\dt}{4}\right)^2}\,
      \left( w^n -\frac{\dt}{4} g\theta \left(\chi^n + \frac{\pi^n_z}{\Gamma g}\right)
      \right)
\\
  & \equiv 
    & \frac{1}{1 + \left(\frac{N\dt}{4}\right)^2}\,
      \left( w^n -\frac{\dt}{4} g\theta\,\chitilde^n
      \right)
\end{eqnarray}
%
With this we find, neglecting the error term for the moment, and with an 
obvious abbreviation for the vertical $\chi$ derivative,
%
\begin{eqnarray}
P_j^{n+\quart*} \left(\chitilde_j^{n+\half} - \chitilde_j^{n}\right)
  & =  
    & - \frac{\dt}{4}\left((P w)_{j+\half}^{n+\quart} + (P w)_{j-\half}^{n+\quart}\right) 
        \left.\pp{\chi}{z}\right|_j^{n+\quart}     
      \\ 
  & =  
    & - \frac{\dt}{4}
        \left(
        \frac{P_{j+\half}^{n+\quart} w_{j+\half}^{n}}{1 + \left(\frac{N_{j+\half}\dt}{4}\right)^2} 
      + \frac{P_{j-\half}^{n+\quart} w_{j-\half}^{n}}{1 + \left(\frac{N_{j-\half}\dt}{4}\right)^2}
        \right) 
        \left.\pp{\chi}{z}\right|_j^{n+\quart}     
      \\ 
  &
    & - \left(\frac{N^*_j \dt}{4}\right)^2  
       \left(
        \frac{P_{j+\half}^{n+\quart} \chitilde_{j+\half}^{n}}{1 + \left(\frac{N_{j+\half}\dt}{4}\right)^2} 
      + \frac{P_{j-\half}^{n+\quart} \chitilde_{j-\half}^{n}}{1 + \left(\frac{N_{j-\half}\dt}{4}\right)^2}
        \right)  
\end{eqnarray}
%
where
%
\begin{equation}
\left(N_{j*}\right)^2 = - g\theta\left.\pp{\chi}{z}\right|_j^{n+\quart}
\end{equation}
%
and where, on the left hand side, we have replace the $\chi$-difference by
the $\chitilde$-difference by adding a zero.

Thus, what we obtain is
%
\begin{eqnarray}
\chitilde_j^{n+\half} 
  & =  
    & \chitilde_j^{n} 
      - \frac{\left(\frac{N^*_j \dt}{4}\right)^2}{P_j^{n+\quart*}}  
        \left(
        \frac{P_{j+\half}^{n+\quart} \chitilde_{j+\half}^{n}}{1 + \left(\frac{N_{j+\half}\dt}{4}\right)^2} 
      + \frac{P_{j-\half}^{n+\quart} \chitilde_{j-\half}^{n}}{1 + \left(\frac{N_{j-\half}\dt}{4}\right)^2}
        \right)  
      \\ 
  &
    & - \frac{\dt}{2} \frac{1}{2 P_j^{n+\quart*} }
        \left(
        \frac{P_{j+\half}^{n+\quart} w_{j+\half}^{n}}{1 + \left(\frac{N_{j+\half}\dt}{4}\right)^2} 
      + \frac{P_{j-\half}^{n+\quart} w_{j-\half}^{n}}{1 + \left(\frac{N_{j-\half}\dt}{4}\right)^2}
        \right) 
        \left.\pp{\chi}{z}\right|_j^{n+\quart}     
\end{eqnarray}
%
The result does look like a complex variant of the explicit part of the
implicit trapezoidal rule, although with weirdly weighted old time
level data entering. 

Of interest is the limit behavior as $\dt\to\infty$ with everything
else fixed. What we obtain is
%
\begin{equation}
\chitilde_j^{n+\half} \to  \chitilde_j^{n} - 2 \chitilde_j^{n*}
\end{equation}
%
where
%
\begin{equation}
\chitilde_j^{n*} 
= \frac{1}{2}\frac{N^2_{j*}}{P_j^{n+\quart*}}
  \left(\frac{ P_{j+\half}^{n+\quart}}{N^2_{j+\half}}\chitilde^n_{j+\half}
      + \frac{P_{j-\half}^{n+\quart}}{N^2_{j-\half}}\chitilde^n_{j-\half}
  \right)\, .
\end{equation}
%

% ===========================================================================

\subsubsection{Oscillator with slow forcing over a two-step cycle?}
\label{sssec:SlowlyForcedOscillatorTwoStepCycle}

Here we consider 
%
\begin{eqnarray}
y^{n+1} = y^n + \dt v^{n+\half}
\end{eqnarray}
%

\newpage

% ===========================================================================
% ===========================================================================
% ===========================================================================

\section{Todos}

\begin{itemize}

\item advect only perturbations of $1/\theta$
  \begin{itemize}
  \item modify density flux accordingly by superimposing reconstructed
        theta-perturbations and background values at cell faces, or
  \item by advecting theta-perturbations as a separate scalar and then
        recomputing full density after the full advection cycle.
  \item[\red{\textbullet}] \red{Tried this, it doesn't improve things, and is
       also somewhat against our philosophy. So, let's forget about this.}
  \end{itemize}

\item Apply explicit part of first-order backward Euler of the pressure-gravity
      combo to $P$-fluxes, but that of the implicit trapezoidal rule to the 
      cell-centered momenta.
  \begin{itemize}
  \item try just using the current implicit variant on both forward steps -- \dgreen{looks good}
  \item fluxes -- \dgreen{produces controlled updates for very large time steps}
  \item cell-centered momenta -- \blue{partial success: vertical velocity gathers 
        oscillatory but controlled updates, yet the buoyancy does not undergo the
        flip-flop in time that would be expected under the implicit trapezoidal rule
        for large time steps. Reason: advection of background theta is done with the
        first-order updated $P$-fluxes. I have to pull this part out.}
  \item pull background-$\theta$ out of density flux and place it back in a source-term
        like expression. -- \red{This is not appropriate because then I begin to handle
        the density flux not consistently with the fluxes of other density-weighted 
        variables. That should be avoided.}
  \item What about going back to separately computing $\tilde S$, i.e., fluctuations of
        $1/\theta$ only for buoyancy computations and resynchronizing after each step?
  \item[\red{\textbullet}] \red{Meanwile tested applying backward Euler half time step
       for the $P$-fluxes in the first OpSplit cycle and forward Euler in the second.
       This formally should yield second order in the end as it would be the equivalent
       of the implicit midpoint rule, and it does give the most favorable results so far.}
  \end{itemize}

\item 2016.12.01: I seem to be missing some part of the $\theta$-update in calculating
      buoyancy for the final momentum update. Options:
  \begin{enumerate}
  \item Add a final $\theta$-buoyancy contribution when the second projection is done.\\
        \red{I tried this: It does improve the asymmetries in the advected gravity wave
        test case, but it introduces unfavorable behavior in the non-advected case. 
        For rather large time steps, checkerboard modes kick in. I think that the 
        asymmetry arising with mean advection has to do with how I link the advection
        split steps (especially the horizontal ones) to the first projection. \\
        Yet, I still have a suspicion that we cannot keep the implicit gravity discretization
        out of the second projection entirely. So, this is where I stand as of August 18, 2017.}
  \item Memorize all pressure-gradient / gravity terms over the predictor cycle and
        recalculate them after the first projection based on the old and new time level
        $\theta$s and on the nodal pressure. This would effectively provide a true
        trapezoidal rule implementation of these terms albeit with a separate update,
        done in the predictor and first projection, for $\theta$.
  \item We get an interesting variant of this approach when we go back to the nodal-pressure
        only version of the flux divergence control. KEEP THIS IN MIND!
  \item Run scheme as is right now, but flip the $x$- and $y$-sweeps in the predictor. 
        Then we get buoyancy evaluated at the beginning and true end of the $x$-sweep
        of advection. In the end, this is a switch from a midpoint to a trapezoidal 
        discretization. Why should this make a difference? Unclear. Also, we would have
        to implement the decomposition of one semi-implicit predictor into two steps, 
        which is somewhat of an arbitrary procedure.
  \end{enumerate}
 
\item 2016.12.01: The errors in the current version appear to be too large. Before I 
      try modifications that seem all to be variants of a second-order accurate 
      scheme, should I not first hunt for a bug that destroys second order in the first
      place?

\end{itemize}

In \eq{eq:deltaw2} we use the abbreviation $\Gamma g \chi^* = - \pi^{\rm hy}_z$
to obtain
%
\begin{eqnarray}
\label{eq:deltaw3a}
\delta w 
  & = 
    & \frac{1}{1 + \Nscsq} 
      \left(- \tau \frac{\theta}{\Gamma}\left(\pi^*_z - \pi^{\rm hy}_z\right)
            - w^* \Nscsq
      \right)
      - \tau \frac{\theta/\Gamma}{1 + \Nscsq} \delta\pi_z 
      \\
\delta w 
  & = 
    & \delta^{\rm expl} w + \delta^{\rm \pi} w
      \\
\delta \chi
  & = 
    & - \tau\, (w^* + \delta^{\rm expl} w) \frac{d\overline{\chi}}{dz} 
      + \frac{1}{\Gamma g}\frac{1}{1 + \Nscsq} (\tau)^2  g\theta\frac{d\overline{\chi}}{dz} \delta\pi_z 
      \\
\delta \chi
  & = 
    & - \tau\, (w^* + \delta^{\rm expl} w) \frac{d\overline{\chi}}{dz} 
      + \frac{1}{\Gamma g}\frac{\Nscsq}{1 + \Nscsq}\delta\pi_z 
\end{eqnarray}
%
It is clear how to implement this in a split fashion outside of the advection
routines. It needs to be implemented also, however, in the flux calculations, 
and it is less clear how that could be achieved. 

\newpage

% ===========================================================================
% ===========================================================================
% ===========================================================================

\section{Preconditioning}
\label{sec:Precon}

\red{So far, this is a side issue. It looks as if the code converges very
rapidly when all the components fit well together, even with vanilla
BiCGStab. We can get back to this when we have the basic scheme running
well. }

% ===========================================================================
% ===========================================================================

\subsection{Introductory thoughts}
\label{ssec:PreconIntro}

Let me recall:  Given $A \in \reals^{n\times n}$, $b \in \reals^n$ find 
$x \in \reals^n$ such that
%
\begin{equation}
A x = b
\end{equation}
%
Preconditioning from the left means multiplying by a matrix 
$B^{-1} \in \reals^{n\times n}$ and considering the problem
%
\begin{equation}
(B^{-1}A) x = B^{-1}b
\end{equation}
%
with the hope that $B^{-1}A$ is better conditioned than $A$. 

Using this strategy, a solver will remain unchanged except for 
%
\begin{enumerate}
\item Computation of the rescaled right hand side $b \to B^{-1}b$ and
\item Extending the call to $Ax$ by the composition of calls $B^{-1}\circ A\, x$. 
\end{enumerate}
%
An interesting question besides this more or less trivial operation 
concerns the computation of the stopping criterion for an iterative
scheme. In some sense, lacking the exact solution, one will want to
use information on the residual to assess how close one has come to 
the solution of the problem. 

Thus, there are two alternatives, 
%
\begin{equation}
\parallel b - Ax^n \parallel < \eps
\qquad\text{or}\qquad
\parallel B^{-1}b - (B^{-1}A)x^n \parallel < \eps \, .
\end{equation}
%
What are the implications? 

To get an intuition, let's take the standard case in meteorology where
we need to compute a solution in a deep atmosphere, across which the
density falls off by a factor $10^{-4}$ or an even more extreme one. 
In the present implementation that solves the 
$P \equiv \overline{\rho\theta}$ equation implicitly and uses an 
Exner type pressure variable, $x \equiv \delta\pi$, the matrix $A$ is 
dominated by the discrete approximation of
%
\begin{equation}
\nabla\cdot\left(\frac{P\theta}{\Gamma} \nabla \delta\pi\right)
\sim 
\nabla\cdot\left(\rho\theta^2 \nabla \delta\pi\right)
\end{equation}
%
and the right hand side is 
%
\begin{equation}
b = \nabla\cdot (P\vv)^*\,,
\end{equation}
%
where the $(\ )^*$ superscript denotes the predicted values before the
implicit divergence controlling step. The divergence constraint 
in the pseudo-incompressible limit reads $\nabla\cdot (P\vv) = 0$, but
the quantity of real interest is the velocity field. Therefore, we 
would want the factor of $P$, which has a dynamic range comparable to the
density, to be scaled out of the div-control. This means we would want 
to control
%
\begin{equation}
{\rm diag}(P)^{-1} \left(b - Ax\right) < \ \parallel {\rm diag}(P)^{-1}\parallel\ \eps\, .
\end{equation}
%
Since division by the diagonal (or the main culprit for extreme dynamic
range in the diagonal) corresponds simply to diagonal preconditioning, 
the answer to the question raised above should be:  Control the residual
of the preconditioned problem,
%
\begin{equation}
\parallel B^{-1} (b - A) x^n \parallel\  < \ 
\parallel B^{-1}\parallel\ \eps \, ,
\end{equation}
%
and not the raw residuum.

% ===========================================================================
% ===========================================================================

\subsection{Preconditioning w.r.t.\ vertical columns}
\label{ssec:PreconVerticalColumn}

% ===========================================================================

\subsubsection{Columnwise preconditioning for the first projection}
\label{ssec:PreconVerticalColumnFirstProjection}

The pressure stencil for the first projection can be read as follows:
%
\begin{equation}
\begin{split}
\left(\Lopfirst \pi\right)_{i,j} 
  & = \frac{h^x_{i+\half,j}\left(\pibar^y_{i+1,j} - \pibar^y_{i,j}\right)
          - h^x_{i-\half,j}\left(\pibar^y_{i,j} - \pibar^y_{i-1,j}\right)}{dx^2}
    \\[10pt]
  & + \frac{h^y_{i,j+\half}\left(\pibar^x_{i,j+1} - \pibar^x_{i,j}\right)
          - h^y_{i,j-\half}\left(\pibar^x_{i,j} - \pibar^x_{i,j-1}\right)}{dy^2}
\end{split}
\end{equation}
%
where 
%
\begin{equation}
\begin{split}
\pibar^x_{i,j}
  & = \frac{\alpha}{2} \pi_{i-1,j} + (1-\alpha)\, \pi_{i,j} + \frac{\alpha}{2} \pi_{i+1,j} 
    \\[10pt]
\pibar^y_{i,j}
  & = \frac{\alpha}{2} \pi_{i,j-1} + (1-\alpha)\, \pi_{i,j} + \frac{\alpha}{2} \pi_{i,j+1} 
\end{split}
\end{equation}
%
The dominant vertical derivative part of the operator thus reads
%
\begin{equation}
\begin{split}
\left(\Lopfirst \pi\right)_{j} 
  & = \frac{h^y_{i,j+\half}\pibar^x_{i,j+1} 
          - \left(h^y_{i,j+\half} + h^y_{i,j-\half}\right) \pibar^x_{i,j}
          + h^y_{i,j-\half}\pibar^x_{i,j-1}}{dy^2}
\end{split}
\end{equation}
%
where 
%
\begin{equation}
\begin{split}
\pibar^x_{i,j}
  & = \frac{\alpha}{2} \pi_{i-1,j} + (1-\alpha)\, \pi_{i,j} + \frac{\alpha}{2} \pi_{i+1,j} \,.
\end{split}
\end{equation}
%
To solve the equation
%
\begin{equation}
\left(\Lopfirst \pi\right)_{j} = r_j
\end{equation}
%
we can obviously first solve for $\pibar$ using the Thomas Algorithm
for the $j$-direction, and then for $\pi$ by inverting the $x$-averaging -- 
again using the Thomas Algorithm. 


\newpage

% ===========================================================================
% ===========================================================================
% ===========================================================================

\section{Hydrostatic initialization}
\label{sec:HydroInit}

The semi-implicit part of the time integration scheme that we intend to 
use is either implicit trapezoidal or implicit midpoint. In both cases
we have an explicit contribution (Euler forward) followed or preceded by
an implicit one (Euler backward). Especially with my implementation up
to Jan 2018, the Euler forward is first, and this implies potentially 
very large deviations from the balanced state in the course of the first
few time steps. This is indeed observed. Now, due to the advective 
nonlinearity, these large deviations can leave a heavy imprint on the
later solution, and this must be avoided. 

Here I describe a pressure initialization by the hydrostatic model. This
is a preliminary exercise also on our way towards constructing a blended
hydrostatic/nonhydrostatic solver. 

The linearized hydrostatic pseudo-incompressible equations for a vertical
slice read
%
\begin{subequations}
\begin{eqnarray}
\label{eq:HorizontalMomentum}
(P\vu)_t + \frac{P\theta}{\Gamma} \pi_x
  & = 
    & 0
      \\
\label{eq:Hydrostatics}
\frac{\theta}{\Gamma}\pi_z
  & = 
    & -  g 
      \\
\theta_t
  & = 
    & - w \frac{d\overline{\Theta}}{dz} 
      \\[5pt]
\label{eq:DivConstraint}
(P\vu)_x + (PW)_z
  & =
    & 0
\end{eqnarray}
\end{subequations}
%
Integrate \eq{eq:DivConstraint} from $z=0$ to $z=h$ and then from 
$0$ to $x$, use the rigid lid conditions on $w$, take the time derivative,
let $\pi = \pi_0(t,x) + \widetilde\pi(t,x,z)$, where
%
\begin{equation}
\widetilde\pi(t,x,z) = - \int\limits_0^z \frac{\Gamma g}{\theta(t,x,\zeta)} \, d\zeta
\end{equation}
%
and obtain
%
\begin{equation}
\int\limits_0^h (P\vu)_t\, dz 
= \left\langle (PU)_t \right\rangle(t)
= - \left\langle\frac{P\theta}{\Gamma}\right\rangle \pi^0_x - 
  \left\langle\frac{P\theta}{\Gamma}\widetilde\pi_x \right\rangle \,.
\end{equation}
%
This yields
%
\begin{equation}
\pi^0_x
= - \frac{\left\langle (PU)_t \right\rangle }{\left\langle P\theta/\Gamma\right\rangle} 
  - \frac{\left\langle P\theta\widetilde\pi_x/\Gamma \right\rangle}%
         {\left\langle P\theta/\Gamma\right\rangle}\,.
\end{equation}
%
Integration in $x$ yields the bottom pressure $\pi^0$ when 
$\left\langle (PU)_t \right\rangle$ is adjusted so as to satisfy the 
pressure boundary conditions, e.g., periodic ones.

% ===========================================================================
% ===========================================================================
% ===========================================================================

\bibliographystyle{FluidMechanics}
\bibliography{Bibliography}

\end{document}